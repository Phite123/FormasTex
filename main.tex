\documentclass[12pt]{amsart}

%Fuente
\usepackage[utf8]{inputenc}
\usepackage[spanish]{babel}
\usepackage[T1]{fontenc}
\usepackage{lmodern}
\usepackage{epigraph}
\usepackage{lipsum}
\usepackage[a4paper, margin=2.5cm]{geometry}
\usepackage{multicol}
\usepackage{amsmath}
\usepackage{enumerate}
\usepackage[shortlabels]{enumitem}
\usepackage{listings}
\DeclareMathOperator{\mcd}{mcd}

\renewcommand{\refname}{Bibliografía}

%diagramas
\usepackage{tikz-cd}

%enumerar páginas y encabezados
\pagestyle{headings}

%Comandos para escribir C más rápido
\usepackage{mathrsfs}
\newcommand{\CC}{\mathbb{C}}
\newcommand{\NN}{\mathbb{N}}
\newcommand{\QQ}{\mathbb{Q}}
\newcommand{\RR}{\mathbb{R}}
\newcommand{\ZZ}{\mathbb{Z}}
\newcommand{\FF}{\mathbb{F}}
\newcommand{\cont}{\mathcal{C}}

\newcommand{\aaa}{\mathbf{a}}
\newcommand{\bbb}{\mathbf{b}}
\newcommand{\vvv}{\mathbf{v}}

\newcommand{\OO}{\mathcal{O}}

\newcommand{\leg}[2]{\left( \frac{#1}{#2} \right)}
\newcommand{\minimat}[4]{\left(\begin{smallmatrix} #1 & #2 \\ #3 & #4 
\end{smallmatrix}\right)}
\newcommand{\lp}{\left(}
\newcommand{\rp}{\right)}
\newcommand{\lb}{\left|}
\newcommand{\rb}{\right|}
\newcommand{\lc}{\left<}
\newcommand{\rc}{\right>}
\newcommand{\lco}{\left[}
\newcommand{\rco}{\right]}

%dobles implicaciones
\usepackage{csquotes}

%Entornos con estilo 'tipo teorema'
\theoremstyle{plain}
\newtheorem{teo}[equation]{Teorema}
\newtheorem{ej}[equation]{Ejercicio}
\newtheorem{prop}[equation]{Proposición}
\newtheorem{lem}[equation]{Lema}
\newtheorem{coro}[equation]{Corolario}
\newtheorem{defi}[equation]{Definición}
\newtheorem{obs}[equation]{Observación}
\newtheorem*{nota*}{Notación}
\renewcommand*{\proofname}{Demostración}



%Hipervínculos cliqueables, en color azul
\usepackage[colorlinks=true,allcolors=blue]{hyperref}

\title{Entrega de ejercicios de formas cuadráticas}

\author{Pedro Nicolas Peña}

\begin{document}

% \begin{abstract}
%     Buee quien era
% \end{abstract}

\maketitle

\section*{Guía 1}

% 4, 11 (código y output), 13 (código y output), 20 (tres ítems)
% y 25.

% \subsection*{Ejercicio 4}
% Sea $f(x) = x^2 + x + N$ con $N>0$ entero.

% \begin{itemize}
%     \item Si $f(k)$ es primo para todo 
%     $0\leq k \leq \sqrt{\frac{N}{3}}$ entero. Probar que 
%     $h(1-4N) = 1$.
%     \item Si $h(1-4N) = 1$ entonces $f(k)$ es primo para 
%     todo $k=0, \dots , N-2$
% \end{itemize}


\subsection*{Ejercicio 11}
Sea $p = 2^{9689} - 1$ un primo de Mersenne.

\begin{multicols}{2}
\begin{itemize}
    \item Hallar la raíz cuadrada de $-163 \pmod{p}$.
    \item Hallar $a, b \in \ZZ$ tales que $p = a^2 + 163 \cdot 
    b^2$.
\end{itemize}
\end{multicols}

\subsection*{Ejercicio 12}
Sea $p = 313$.

\begin{multicols}{2}
\begin{itemize}
    \item Calcular con buena precisión
    $$u := \prod_{k=1}^{p} \sin^{-\lp \frac{k}{p}\rp} \lp 
    \frac{k\pi}{p} \rp.$$
    \item Hallar $\frac{u + u^{-1}}{2}$.
    \item Resolver en enteros $x^2 - 313y^2 = 1$.
    \item Sea $v = \sqrt{u}.$ Hallar $\frac{v - v^{-1}}{2}$.
    \item Resolver en enteros $x^2 - 313y^2 = -1$.
\end{itemize}
\end{multicols}

\subsection*{Ejercicio 13}
Imitar el método del ejercicio 12 para $x^2 - 1093y^2 = 1$. 
¿Qué sucede?

% \subsection*{Ejercicio 20}
% En cada caso, hallar $Cl (\ZZ[\sqrt{d}])$ y $Cl^+ 
% (\ZZ[\sqrt{d}])$:


% \begin{multicols}{2}
% \begin{itemize}
%     \item $d = 6$,
%     \item $d = 10$,
%     \item $d = 5$,
%     \item $d = 45$,
%     \item $d = 10 \cdot 5^2$,
%     \item $d = 10 \cdot 7^2$.
% \end{itemize}
% \end{multicols}

\subsection*{Ejercicio 25}
Hallar el grupo de unidades de $\OO = \ZZ[\sqrt{17}]$. 
Calcular el orden del grupo de unidades de $\OO/97\OO$. 
Deducir el grupo de unidades de $\ZZ[97\sqrt{17}]$ 
(la compu puede ayudar acá). Resolver en enteros 
$x^2-17\cdot 97^2\cdot y^2=1$.

\section*{Guía 2}

% 4, 8, 12, 16 (un ítem) y 22.

\subsection*{Ejercicio 4}
Sea $q$ regular e isotrópica sobre $k^n$. Probar que hay 
$\aaa, \bbb \in k^n$ tales que $q \lp \aaa + t \bbb \rp = t$.

\subsection*{Ejercicio 8}
Sea $q$ ternaria sobre $\QQ_p$ con $p \geq 3$ primo. 
Supongamos que $|\det(q) |_p = 1 = c_p(q)$. Probar que $q$ es isotrópica.

% \subsection*{Ejercicio 12}
% Sean $a, b, a', b' \in \QQ_p^\times$. Probar que:
% \begin{itemize}
%     \item El núcleo de $x \mapsto \lp x, a \rp_p$ es un subgrupo 
%     de $\QQ_p^\times$ de índice $\leq 2$.
%     \item La ecuación $(x, -ab)_p = (a, b)_p$ tiene solución.
% \end{itemize}

% Concluir que si el sistema
% $
% \begin{cases}
% (x, -ab)_p & = (a, b)_p \\
% (x, -a'b')_p & = (a', b')_p
% \end{cases}
% $
% no tiene solución en $\QQ_p^\times$, entonces $(a, b)_p 
% \neq (a', b')_p$ y $ab a'b' \in \lp \QQ_p^\times \rp^2$.

% \subsection*{Ejercicio 16}
% Probar que el anillo $W(\QQ_p)$ es isomorfo (como anillo) a:
% \begin{enumerate}[a)]
%     \item $(\ZZ / 2\ZZ)[C_2 \times C_2]$ si $p \equiv 1 
%     \pmod{4}$.
%     \item $(\ZZ / 4\ZZ)[C_2]$ si $p \equiv 3 \pmod{4}$.
%     \item $(\ZZ / 8\ZZ)[s, t] / \lp 2s, 2t, s^2, t^2, st - 4 \rp$
%     si $p = 2$.
% \end{enumerate}

% \subsection*{Ejercicio 22}
% Hallar todos los $\lambda \in \QQ$ tales que 
% $\lp \lambda + 1 \rp \lp x^2+y^2+z^2 \rp + \lambda w^2$ y 
% $x^2 + y^2 + z^2 + \lambda \lp \lambda + 1 \rp w^2$
% son equivalentes.

\section*{Guía 3}

% 3 items del 3, 5, 6(b), 8 y 9

\subsection*{Ejercicio 3} Probar las equivalencias de las siguientes
formas cuadráticas sobre $\ZZ_2$:

\begin{enumerate}[a)]
    \item $x^2+y^2\sim 5x^2+5y^2$.
    \item $x^2+2y^2\sim 3x^2+6y^2$.
    \item $x^2+4y^2\sim 5x^2+20y^2$.
    \item $ux^2+2yz\sim ux^2+y^2-z^2$ (con $|u|_2=1$).
    \item $ux^2+(2y^2+2yz+2z^2)\sim u(3x^2-y^2-z^2)$ (con $|u|_2=1$).
    \item $(2x^2+2xy+2y^2)+(2z^2+2zw+2w^2)\sim 2xy+2z$.
\end{enumerate}

En particular, la cuarta relación dice que no hay un Teorema de 
cancelación de Witt general en $\ZZ_2$. Deducir que tampoco lo hay 
en $\ZZ$.

\subsection*{Ejercicio 5}

Probar que $5x^2 + 11y^2$ y $x^2+55y^2$ son $\ZZ_p$-equivalentes
para todo $p$ primo pero no son $\ZZ$-equivalentes.

\subsection*{Ejercicio 6, b}

Determinar los primos $p$ tales que las siguientes formas son
equivalentes en $\ZZ_p$: $5x^2 + 13y^2 + 11z^2 + 2yz + 2xz +
16xy$ con $x^2 + 16y^2 + 10z^2 + 14yz - 4xz + 2xy$.

\subsection*{Ejercicio 8}

\begin{enumerate}[1.]
\item Probar que las formas cuadráticas
    \begin{itemize}
        \item $g(x,y)=3x^2+2xy+23y^2$
        \item $h(x,y)=7x^2+6xy+11y^2$
    \end{itemize}
    estan en el mismo género pero no son equivalentes.
\item Probar que las formas ternarias indefinidas $z^2-g(x,y)$ y 
    $z^2-h(x,y)$ son equivalentes propiamente.

    \textit{Sugerencia: hallar un vector no canonico de norma 1 para 
    la segunda y hacer un cambio de variables.}
\end{enumerate}

\section*{Guía 4}

% del 1 al 5

\subsection*{Ejercicio 2} Sea $q=x^2+xy+3y^2-2z^2$. Describir 
$\mathcal{C}(V,q)$ por generadores y relaciones. Hallar la matriz 
de $\tilde{q}$, la forma cuadrática inducida por $\tilde{q}:=u
\varepsilon (u)$.




\newpage

\section*{Soluciones}

% \subsection*{Guía 1, ejercicio 4}

% Sea $D=1-4N$. Para ver que $h(D)=1$ tenemos que ver que no hay 
% otra forma reducida de discriminante $D$, es decir otra manera 
% de escribir $D=b^2-4ac$ con $|b|\leq|a|\leq|c|$ (y si hay un 
% $=$ entonces $b\geq0$) que no sea $D=1^2-4\cdot 1 \cdot N$. 
% Observemos que $1-4N=b^2-4ac \iff \frac{b^2-1+4N}{4}=ac \iff 
% \frac{b^2-1}{4} + N = ac$.

% Supongamos que $g(x,y)=Ax^2+Bxy+Cy^2$ es una forma cuadrática 
% reducida de discriminante $D$, osea $D=B^2-4AC$ y en particular 
% $|B|\leq|A|\leq\sqrt{\frac{-D}{3}}$. Como $D\equiv 1\mod 4$ 
% entonces $B$ es impar y $\frac{B-1}{2}$ es entero. Observemos 
% también que (con $B>0$) vale $B\leq \sqrt{\frac{-D}{3}} \iff 
% \frac{B-1}{2} \leq \sqrt{\frac{4N-1}{12}}-\frac{1}{2} \leq 
% \sqrt{\frac{4N}{12}}=\sqrt{\frac{N}{3}}$.

% Ahora evaluando en $f$:

% $$f\lp\frac{B-1}{2}\rp = \lp\frac{B-1}{2}\rp^2 + \frac{B-1}{2} 
% + N = \frac{B^2-1}{4}+N$$ 

% Nos quedó que $\frac{B^2-1}{4}+N = AC$ debe ser primo por 
% hipótesis, por lo que $A=1$, $C=N$ y entonces $|B| = 1$ 
% ($B\neq 0$ porque erá impar) pero como son iguales en módulo 
% necesariamente $g(x,y)=x^2+xy+Ny^2$ es la única reducida de 
% discriminante $D$.

% Falta el punto b






\subsection*{Guía 1, ejercicio 11}

El ejercicio 8 nos dice que si tenemos un primo congruente a 
$-1$ módulo $4$ (como lo es nuestro $p=2^{9689}-1$ pues es 
$4\cdot 2^{9687} -1$) y un resto cuadrático $a$ entonces 
$a^{\frac{p+1}{4}}$ es una raíz de $a$ módulo $p$. Veamos que 
$a = -163$ es un resto cuadrático:

$$\leg{-163}{p} =\leg{-1}{p}\leg{163}{p} = (-1)^{\frac{p-1}{2}}
(-1)^{\lp\frac{163-1}{2}\rp\lp\frac{p-1}{2}\rp} \leg{p}{163} =$$
$$= \leg{2^{9689}-1}{163}=\leg{2^{131}-1}{163}$$

En la segunda igualdad usamos la definición de $\leg{-1}{p}$ y 
reciprocidad cuadrática, que nos dan dos $-1$ (pues $p$ y $163$ 
son ambos $3\mod{4}$) que se cancelan y nos dejan con la tarea 
de determinar $\leg{p}{163}$, cuya complejidad pudimos reducir 
por ``fermatito'' ($9689 = 59*162 + 131$) pero sigue siendo 
grande así que vamos a buscar un número congruente a $2^{131}-1$ 
módulo $163$ con la computadora, el \hyperref[codigo1]{código} 
nos dice que $2^{131}-1\equiv 74 \mod{163}$ y nos queda 
encontrar $\leg{74}{163}$:

$$\leg{74}{163} = \leg{2}{163}\leg{37}{163} = (-1)^{\frac{163^2-1}{8}}
\leg{163}{37}= (-1) \leg{15}{37} =$$
$$= (-1)\leg{3}{37}\leg{5}{37} = (-1)\leg{1}{3}\leg{2}{5}$$

Con lo que nos queda $(-1)^2$ y resulta que $-163$ era un 
cuadrado módulo $2^{9689}-1$. 

Ahora tenemos que calcular $(-163)^{\frac{2^{9689}-1+1}{4}} 
\equiv 163^{2^{9687}} \mod{2^{9689}-1}$ y a falta de algún 
método simplificador lo metemos en sage 
($\texttt{power\_mod(163, 2\^{}9687, 2\^{}9689-1)}$) y nos
da un número de 2917 cifras que llamaremos $r$.

Tomando un consejo de un sabio, miremos un poco el ejercicio anterior, 
que para encontrar solución a $a^2 +b^2 = p$ nos hace calcular
$d=\mcd(p,c+i)$ para $c$ una raíz de $-1\mod p$. Ayudándonos de
$\overline{d} = \mcd\lp\overline{p},\overline{c+i}\rp = \mcd\lp p,c-i\rp$ 
tenemos $d\cdot \overline{d} |p^2$ y a su vez 
$d\cdot \overline{d} |(c+i)(c-i)=c^2+1 = k\cdot p$ para algún 
$k<p$ pues $c<p$. Luego $d\cdot \overline{d} = p$ (o $1$), y como
$d$ es de la forma $a+bi$, entonces cumple $d\cdot \overline{d}=a^2+b^2 = p$.
Para encontrar el análogo en nuestro ejercicio precisamos 
$d=\mcd \lp p,r+\sqrt{-163} \rp$, que obtenemos con un simple 
\hyperref[codigo2]{código} de sage y nos da la 
\hyperref[solucion2]{solución}.

\subsection*{Guía 1, ejercicio 13}

El enunciado sugiere que hablemos un poco del ejercicio 12 
previamente. Una posible solución es la de calcular el valor 
que pide el punto $a$, redondear al entero siguiente y dividir 
por 2 como sugiere el punto $b$, meter ciegamente ese número en 
el $x$ de la ecuación del $c$ y chequear si el $\frac{x^2-1}
{313}$ es un cuadrado (que efectivamente lo es), y llamándolo 
$y^2$ encontramos $x^2-313y^2=1$. También estaría bueno saber 
por que funcionó.

Con la fórmula de $\sen(\alpha) = \frac{e^{i\alpha}-
e^{-i\alpha}}{2i} = \frac{e^{-i\alpha}(e^{2i\alpha}-1
)}{2i}$ tenemos que:

$$\prod^{p-1}_{k=1} \sen\lp\frac{k\pi}{p}\rp
^{-\lp\frac{k}{p} \rp} = \prod^{p-1}_{k=1} \lp e^{-\frac
{ik\pi}{p}} - e^{-\frac{ik\pi}{p}}\rp^{-\lp
\frac{k}{p} \rp}= \prod^{p-1}_{k=1} \lp e^{\frac
{ik\pi}{p}}\rp^{-\lp \frac{k}{p} \rp} \prod^{p-1}_{k=1}
\lp e^{\frac{2ik\pi}{p}}-1\rp^{-\lp\frac{k}{p} \rp}$$

pues como la mitad son cuadrados y la otra no entonces:

$$\prod^{p-1}_{k=1} \frac{1}{2i} ^ {-\lp \frac{k}{p} \rp} = 1$$

Entonces $u$ es una unidad pues $\prod^{p-1}_{k=1} \lp 
\xi_p^k-1\rp^{-\lp\frac{k}{p} \rp}$ tiene norma $1$ por el 
teorema 90 de Hilbert (es un cociente de elementos conjugados
por el grupo de Galois), y el otro factor es una potencia de 
una unidad por lo que $u$ es produto de unidades. Pero esta
unidad vive en un lugar particular. Notemos que al aplicar los 
automorfismos de Galois $\sigma_j: \xi_{p}^n \to \xi_{p}^{jn}$
(como $\QQ(\xi_p) = \QQ(\xi_{2p})$ estos son todos),
se factorizan por el producto, si $j$ es un cuadrado dejan
fijo a $u$ y sinó lo invierten:

$$\sigma_a(u) = \prod_{k=1}^{p-1} \sigma_a \lp
(\xi_{2p}^k - \xi_{2p}^{-k})^{\lp\frac{k}{p}\rp}
\rp = \prod_{k=1}^{p-1} \lp\sigma_a \lp\xi_{2p}^k\rp
- \sigma_a \lp\xi_{2p}^{-k}\rp\rp^{\lp\frac{k}{p}\rp} = $$

$$=\prod_{k=1}^{p-1} \lp\xi_{2p}^{ak}
- \xi_{2p}^{-ak}\rp^{\lp\frac{k}{p}\rp}=\prod_{k=1}^{p-1}
\lp\xi_{2p}^{k} - \xi_{2p}^{-k}\rp^{\lp\frac{a}{p}\rp} = 
u^{\lp\frac{a}{p}\rp}$$

Entonces $u$ vive en el subcuerpo que fijan los $\sigma_a$ con
$a$ un resto cuadrático$\mod p$, que es el que queremos: 
$\QQ(\sqrt{p})$, y tambien es una unidad allí pues lo era en
el cuerpo más grande. Y $u^{-1}$ pertenece a $\QQ(\sqrt{p})$
por la misma razón y aparte cumple que $u^{-1}=\bar{u}$ (la
norma en $\QQ(\sqrt{p})$ es $u\bar{u}=1$). Si $u=x+\sqrt{p}y$
entonces $u^{-1}=x-\sqrt{p}y$ y el producto tiene la forma 
que queremos. Y luego $u+\bar{u}=2x$ es la forma de encontrar
la solución a la ecuación $x^2-py^2=1$. 

No necesariamente $u$
nos queda una unidad fundamental de $\QQ(\sqrt{p})$ y esto
va a ser importante para el siguiente ejercicio. Nos falta 
decir por qué $2x$ es un entero, pero es suma de 
enteros algebraicos y aparte queda fijo por todos los 
automorfismos de Galois, asi que tiene que serlo (ojo, no 
necesariamente $x$ es entero).

Por otro lado, en el ejercicio 13, cuando calculamos para 
$p=1093$ el $u$:

$$u= \prod^{1092}_{k=1} \sen\lp\frac{k\pi}{1093}\rp ^{-\lp
\frac{k}{1093} \rp}$$

nos queda (casi) impar $u = 1545688332149050.999999999999999353
\cdots$, sabemos que $u+u^{-1} = 1545688332149051$. Esto nos 
dice que la solución a la ecuación que obtuvimos es:

$$\lp\frac{1545688332149051}{2}\rp^2 - 1093 
\lp\frac{46753254507927}{2}\rp^2 = 1$$

(despejando el segundo valor). Esto no nos dio una solución 
en números enteros y tampoco nos dio la más pequeña posible.

Primero busquemos la más chica: sabemos que si $u$ es unidad 
en $\QQ(\sqrt{1093})$, entonces tiene que ser una potencia de 
la unidad fundamental pues $\OO_{\QQ(\sqrt{p})}$ tiene 
dimensión 1; y podemos buscar esta potencia numericamente.
Con un simple \hyperref[codigo3]{código} encontramos que $u$
es potencia quinta de la solución fundamental, y tenemos 
la siguiente solución:

$$\lp\frac{1091}{2}\rp^2 - 1093 \lp\frac{33}{2}\rp^2 = 1$$

Como las soluciones son potencias de la más pequeña,
elevando al cubo $\frac{1091}{2} + \sqrt{1093}
\frac{33}{2}$ (esto siempre funciona para sacar los 
denominadores) nos queda:

$$\lp\frac{1091}{2}\rp^3 +3\cdot1093
\lp\frac{1091}{2}\rp \lp\frac{33}{2}\rp^2 + \sqrt{1093} 
\lp 3\lp\frac{1091}{2}\rp^2\lp\frac{33}{2}\rp + 1093
\lp\frac{33}{2}\rp^3\rp$$

y viendo módulo 8 nos quedan enteras las coordenadas, es decir
encontramos la solución a la ecuación de Pell que buscabamos:

$$649296649^2 - 1093\cdot19639620^2 = 1$$

Si no hubiesemos querido la menor solución podriamos haber 
encontrado tambien una con el $u$ que habíamos encontrado
elevandolo al cubo, y esta sería:

$$1846442509769358115211770346218978772257666249^2 - $$
$$1093\cdot 55850325578557669515530976378442112225078100^2 = 1$$


% \subsection*{Guía 1, ejercicio 20}

% Para cada $d$ precisamos el discriminante $D$ del cuerpo 
% $\QQ[\sqrt{d}]$ que será igual a $d'$ si $d'\equiv 1\mod 4$ o 
% $4d$ si $d' \equiv 2,3 \mod 4$ para $d'$ la parte libre de 
% cuadrados de $d$. Luego tambien es util conocer el anillo de 
% enteros de ese cuerpo, que siempre es $\OO_k = 
% \ZZ\lco\frac{D+\sqrt{D}}{2}\rco$, pues todo orden se puede escribir como 
% $\OO = \ZZ + f \OO_k$ y el discriminante de $\OO$, que será
% $f^2 D$ para el $f$ mínimo, lo determina univocamente.

% Para calcular el orden de $Cl(\ZZ[\sqrt{d}])$ podemos usar 
% la siguiente sucesion exacta:

% $$1\to \OO_k^\times/\OO^\times \to 
% \frac{(\OO_k/\mathfrak{f})^\times}{(\OO/\mathfrak{f})^\times}
% \to Cl(\OO) \to Cl(\OO_k) \to 1 $$

% O como corolario que:

% $$h(\OO) = \frac{h(\OO_k)\cdot \lco (\OO_k/\mathfrak{f})
% ^\times : (\OO/\mathfrak{f})^\times \rco}{\lco \OO_k^\times
% : \OO^\times \rco}$$

% Recordando que si $\OO_k^\times =\{\pm1\} \lc \varphi \rc$ 
% entonces $\OO^\times = \{\pm1\} \lc \varphi^k \rc$ para algun 
% $k$ y con eso $\lco \OO_k^\times : \OO^\times \rco = k$.
% Tambien nos facilita la cuenta que $\lco (\OO_k/\mathfrak{f})
% ^\times : (\OO/\mathfrak{f}) ^\times \rco$ será isomorfo a
% $(\ZZ/\mathfrak{f}\ZZ)^\times$.

% Busquemos los conductores de cada uno de esos órdenes:

% En $\QQ(\sqrt{5})$ el $\OO_k$ es $\ZZ\lco\frac{1+
% \sqrt{5}}{2}\rco$. $\ZZ[\sqrt{5}]$ tiene conductor $2$, 
% $\ZZ[3\sqrt{5}]$ tiene conductor 

% En $\QQ(\sqrt{10})$ el $\OO_k$ es $\ZZ\lco\sqrt{5}\rco$.
% $\ZZ[5\sqrt{10}]$ tiene conductor y $\ZZ[7\sqrt{10}]$ 
% tiene conductor 

% O una prop que nos dice que el cardinal del conjunto de 
% ideales fraccionario inversibles cocientado por la accion de
% $\QQ^\times$ es igual al cardinal de elementos en el siguiente
% conjunto:

% $$\left\{ \tau = \frac{-b+\sqrt{D}}{2a}:a>0, 4a|D-b^2, 
% \mcd(a,b,c)=1 \right\} / \ZZ$$

% ojo que las formas reducidas son solo para disc < 0

% Entonces veamos las formas de escribir $D=b^2-4ac$ con 
% $|b|\leq|a|\leq|c|$ y $\mcd(a,b,c)=1$ para $D=5, 24, 40$:

% $5=1^2 + 4 \cdot 1 \cdot 1$

% $24=0^2+4\cdot1\cdot6 = 0^2 + 4\cdot2\cdot3$

% $40 = 0^2+4\cdot1\cdot10 = 0^2+4\cdot2\cdot5 =
% 2^2+4\cdot3\cdot3$


% Luego para calcular $Cl(\OO)$ veremos si es igual o el doble
% del $Cl^+(\OO)$ debido a que esas son las unicas posibilidades.

% Hay elementos de norma $-1$ en nuestro anillo? 

% en $\ZZ[\sqrt{5}]$ si pues $-1 = 2^2-5$

% en $\ZZ[\sqrt{6}]$ no pues $-1 = x^2- 6y^2$ no tiene 
% solución módulo 3.

% en $\ZZ[\sqrt{10}]$ si pues $-1 = 3^2-10$

% en $\ZZ[\sqrt{45}]$ no, de vuelta por módulo 3

% en $\ZZ[5\sqrt{10}]$ nose xd

% en $\ZZ[7\sqrt{10}]$ no, por módulo 7



\subsection*{Guía 1, ejercicio 25}

Para encontrar un elemento que genere el grupo de unidades de
$\OO=\ZZ[\sqrt{17}]$ primero notemos que $4^2-17\cdot 1=-1$. 
Entonces facilmente tenemos $\OO^\times = \{\pm 1\} \times 
\lc4+\sqrt{17}\rc_\ZZ$. 

Ahora usemos algunos resultados teoricos para facilitar el calculo de 
$\OO / 97\OO$. El anillo de enteros del cuerpo $A=\QQ[\sqrt{17}]$ es 
$\OO_K = \ZZ[\frac{1+\sqrt{17}}{2}]$, por lo que nuestro
orden tiene conductor $2$ y el ideal $97\OO$ es coprimo con ese conductor
por lo que $\OO /97\OO \cong \OO_K / 97 \OO_K$
y ahora podemos ver el comportamiento de $97\OO_K$
en función de si $17$ es un cuadrado módulo $97$:

$$\leg{17}{97} = \leg{97}{17} = \leg{12}{17} = \leg{3}{17} =
\leg{17}{3} = \leg{2}{3} = -1$$

Ahora, como $17$ no es un cuadrado módulo $97$, el ideal 
$97\OO_K$ es un primo (porque nos quedó inerte)
y $\OO/97\OO \cong \OO_K / 97\OO_K \cong\ZZ/97\ZZ$ tiene orden $97$.

El grupo de unidades de $\ZZ[97\sqrt{17}]$ será generado por
una potencia del de $\ZZ[\sqrt{17}]^\times$. Ayudandonos de la 
computadora para encontrar una que tenga segunda coordenada divisible
por $97$ iteramos $\verb|Zsqrt17((4+a)^i)|$ hasta obtener que 
$(4+\sqrt{17})^{49}$ tiene el coeficiente que acompaña a $\sqrt{17}$
divisible por $97$ (¿tal vez era obvio que iba a ser $\frac{97+1}{2}$?)
por lo que tiene que generar las unidades de $\ZZ[97\sqrt{17}]$.

El número obtenido tiene norma $-1$, porque tambien la tenía el 
$4+\sqrt{17}$, por lo que para obtener una solución a 
$x^2+17\cdot97^2\cdot y^2 = 1$ todavía necesitamos elevar el generador
al cuadrado. 
\hyperref[solucion25]{$(4+\sqrt{17})^{98}\cdot(4-\sqrt{17})^{98} = 1$} 
y será de la forma deseada.

\subsection*{Guía 2, ejercicio 4}

Si tenemos una forma isotropica no degenerada significa que en el
espacio cuadratico $(V,q)$ tenemos un plano de hiperbolico incluido. 
Esto es si $q(v)=0$ con $v\neq0$ entonces usando $v$ en la primera 
columna de una base tengo un $0$ en la primera coordenada de la forma
vista en esa base. Despues existirá un $w$ tal que $\phi(v,w)\neq0$
pues $q$ era regular, y reescalando tenemos $\lco\phi\rco_B=
\minimat{0}{1}{1}{*}$ sobre las primeras coordenadas. Para eliminar 
el $*$ notemos que habrá un $\lambda$ tal que $q(w-\lambda v)=0$ 
(es restar multiplos de la primera fila y columna que tienen un 1).

Completando a una base, tenemos $B$ tal que $\lco\phi\rco_B=
\minimat{0}{1}{1}{0}$ en su primer bloque de $2\times2$ y con eso 
$q'(\aaa+t\bbb)=t$ para $\aaa=(0,1/2,0,0,\cdots)$ y $\bbb=(1,0,0,
\cdots)$ donde $q'$ fue ver la forma en la base $B$. Para que esto 
sea general para $q$ tenemos que:
$$t=(\aaa +t\bbb )^T \lp C_B^T\lco\phi\rco C_B\rp(\aaa+t\bbb)=
\lp C_B(\aaa+t\bbb)\rp^T\lco\phi\rco \lp C_B(\aaa+t\bbb)\rp$$
y entonces con $\aaa'=C_B\aaa$ y $\bbb'=C_B\bbb$ tenemos $q(\aaa'+
t\bbb')=t$.

Observacion: asumimos que la dimension de $q$ era mayor o igual a 2
pues la unica forma isotropica de dimension 1 no es regular.

\subsection*{Guía 2, ejercicio 8}

Queremos ver que $q = \lc a,b,c\rc $ con $a,b,c \in \QQ_p$ es 
isotrópica si $|\det (q)|_p =1$ y $C_p(q) = 1$.

Usaremos la propiedad de la teorica que nos dice que para ternarias,
$q$ isotropica si y solo si $C_p(q)=(-1,-\det (q))_p$. La vuelta se 
demuestra viendo que si reescalamos $\lc a,b,c\rc \sim \lc\lambda a,
\lambda b,\lambda c\rc \sim \lc-a/c,-b/c,-1\rc$ preserva isotropia, 
y la ultima es isotrópica si y solo si $(-a/c,-b/c)_p = 1$. 
Esta condicion es equivalente a la que pediamos porque:
$$C_p(\lc a,b,-1\rc) = (a,b)(-1,a)(-1,b)=(a,b)(-1,ab)=(a,b)
(-1,-\det(q))$$ son iguales solo si $(a,b)=1$.

Al reescalar, el invariante de Hasse y el valor de $(-1,-\det(q))_p$ 
cambian de la siguiente forma:
$$C_p(\lambda q) = C_p(q) (\lambda,\lambda)_p^3 = C_p(q) 
(\lambda, -1)_p$$
$$(-1,-\det(\lambda q))_p = (-1,-\lambda^3\det(q))_p=
(-1,\lambda)(-1,-\det(q))_p$$
es decir, cambian por el mismo factor de $(-1,\lambda)$ entonces siguen
siendo iguales si antes lo eran.

Llamemos $d = -\det (q)$. Con esto, $q$ isotrópica si y solo si 
$(-1,d)_p =1$, que es lo mismo que $(d,d)_p =1$, y esto vale
pues $|d|_p =1$ y la ecuación $d x^2 + d y^2 = 1$ en 
$\FF_p$ siempre tiene solución ($d\neq 0$ en $\FF_p$), y
podemos llevarla a $\QQ_p$ por el lema de Hensel.

\subsection*{Guía 3, ejercicio 3} Vamos a hacer cuentitas. Para
probar las equivalencias vamos a ``patear el determinante'' al final
y probar que las formas que nos dan son equivalentes a $x^2+dy^2$.

$5x^2+5y^2$: Si multiplicamos $\minimat{5}{0}{0}{5}$ por 
$\minimat{a}{b}{c}{d}$ y su transpuesta encontramos que queremos 
resolver el siguiente sistema de ecuaciones:
$$\left\{
\begin{matrix}
    5a^2+5c^2=5b^2+5d^2 &=&1 \\
    5(ab+cd) &=&0 \\
    ad-bc &=&1 \\
\end{matrix}\right.$$

La ecuacion $5x^2+5y^2\equiv 1\mod 8$ tiene como solución cualquier
$(x,y)$ donde uno sea par y el otro impar, y esto ayuda a encontrar 
la solución. Notamos rapidamente que no podemos poner la misma 
solución dos veces pues la primera nos exige que el elemento par sea
$2\mod4$ mientras la segunda precisa $0\mod4$ para tener solución 
módulo 8. Con esa info probamos: si $a=2$ y $c=1$ entonces 
precisamos $-2b=d$ y para el determinante nos queda $b=-\frac{1}{5}$.
Con esto tenemos:
$$\lp\begin{matrix}
    2&1 \\
    -\frac{1}{5}&\frac{2}{5} \\
\end{matrix}\rp
\lp\begin{matrix}
    5&0 \\
    0&5 \\
\end{matrix}\rp
\lp\begin{matrix}
    2&-\frac{1}{5} \\
    1&\frac{2}{5} \\
\end{matrix}\rp = 
\lp\begin{matrix}
    25&0 \\
    0&1 \\
\end{matrix}\rp$$
que nos da $5x^2+5y^2\sim_{\ZZ_2} 25x^2+y^2 \sim_{\ZZ_2} x^2+y^2$ 
con la ultima usando que $5\in \ZZ_2^\times$.

$3x^2+6y^2$: Como antes solo que ahora con $3$ en vez de $5$ y con un 
$2$ extra en la segunda coordenada. Lo primero cambia poco en el 
razonamiento mientras lo segundo es mas relevante, pero igualmente 
escribimos el sistema:
$$\left\{
\begin{matrix}
    3a^2+6c^2 &=&1 \\
    3b^2+6d^2 &=&2 \\
    3ab+6cd &=&0 \\
    ad-bc &=&1 \\
\end{matrix}\right.$$

En la segunda ecuacion, poniendo $b=2$ y $c=1$ nos queda $=18$ por 
lo que dividiendo por 9 que es un cuadrado tenemos $b=\frac{2}{3}$
y $c=\frac{1}{3}$, con eso por la tercera sabemos $a=-c$ y la cuarta 
nos obliga a que sean $a=1$ y $c=-1$. Con esto nos queda $3x^2+6y^2
\sim 9x^2+2y^2\sim x^2+2y^2$ como queriamos ver.

$ux^2+y^2-z^2\sim ux^2+2yz$: Evidentemente no podemos dejar fija la
coordenada $x$ y mover las otras. En vez de poner las letras de la $a$
a la $i$ veamos haciendo cambios inteligentemente sobre las formas:

Primero podemos cambiar en la primera $y\to y+z$ de modo que 
$ux^2+(y+z)^2-z^2=ux^2+y^2+2yz$ que ya nos deja el término $2yz$ que 
necesitamos. Ahora probando encontramos los siguientes cambios 
convenientes: $x\to x+\alpha y$ nos deja $ux^2+2u\alpha xy +
(u\alpha^2+1)y^2+2yz$ y luego $z\to z+\beta x+\gamma y$ queda
$ux^2+2yz+2(\beta+\alpha u)xy+(2\gamma+\alpha^2u+1)y^2$ que puede 
darnos una solución.

Quiero $\beta+\alpha u=0$ y $2\gamma+\alpha^2u+1=0$. La ultima nos 
dice que $\alpha=\pm\sqrt{\frac{-1-2\gamma}{u}}$, y para que este 
elemento exista en $\ZZ_2$ tenemos que tener $\frac{-1-2\gamma}{u}
\equiv 1\mod 8$, es decir $-1-2\gamma\equiv u \iff 2\gamma\equiv 
1-u$ que tendrá solución debido a que como $u$ era una unidad 
tiene que ser impar, y luego $1-u$ es divisible por $2$. Con eso 
despejamos $\gamma$ y $\alpha$, aparte nos queda $\beta=-\alpha u$
y juntando todo tenemos:

$$\lp\begin{matrix}
    u&0&-u\sqrt{\frac{-1-2\gamma}{u}} \\
    \sqrt{\frac{-1-2\gamma}{u}}&1&\gamma \\
    0&0&1 \\
\end{matrix}\rp
\lp\begin{matrix}
    u&0&0 \\
    0&1&1 \\
    0&1&0 \\
\end{matrix}\rp
\lp\begin{matrix}
    1&\sqrt{\frac{-1-2\gamma}{u}}&0 \\
    0&1&0 \\
    -u\sqrt{\frac{-1-2\gamma}{u}}&\gamma&1 \\
\end{matrix}\rp=
\lp\begin{matrix}
    u&0&0 \\
    0&0&1 \\
    0&1&0 \\
\end{matrix}\rp$$

donde sabemos que existe un $\gamma \in \ZZ_2$ tal que 
$\sqrt{\frac{-1-2\gamma}{u}}\in\ZZ_2$. Componiendo las dos 
transformaciones tenemos la equivalencia de $ux^2+y^2-z^2$ a
$ux^2+2yz$.

Observamos que este ejercicio nos muestra que no hay un analogo 
directo del teorema de cancelacion de Witt en $\ZZ_2$ pues nos 
diria que podemos cancelar el $ux^2$ que se encuentra en ambas 
formas, pero en este caso nos quedaría que, como probamos que las formas
completas eran equivalentes, $y^2-z^2$ es equivalente a $2yz$ pero
eso no es cierto en $\ZZ_2$ (la primera puede representar impares).
Si hubiese un teorema de cancelacion de Witt en $\ZZ$ entonces 
podriamos ver los mismos valores en $\ZZ_2$ y ya demostramos que 
ahí no funciona.

\subsection*{Guía 3, ejercicio 5}

Veamos en tres partes: las formas son equivalentes en $\ZZ_5$
y $\ZZ_{11}$, luego en $\ZZ_p$ y luego que no lo son en $\ZZ$.

En $\ZZ_5$ tengo que $11$ es un cuadrado pues podemos levantar
la solución de $x^2 \equiv 11\equiv 1 \mod 5$ por Hensel. Luego en 
ambas formas podemos meterlo en la variable y nos quedan 
$5x^2+y^2$ y $x^2+5y^2$ que son equivalentes.

En $\ZZ_{11}$ tengo que, como $4^2 \equiv 5\mod 11$ 
entonces puedo hacer un razonamiento similar al anterior.

Para cualquier otro $p$ (salvo el innombrable) los números 
$5$ y $11$ son unidades, lo que implica que la ecuación 
$5x^2 + 11y^2=1$ tiene solución en $\FF_p$, y como podemos
representar el $1$ entonces nuestra forma cuadrática es
equivalente a $x^2+ d y^2$ para $d$ el determinante, que 
era $5\cdot 11 = 55$, la otra forma que teniamos. 

En $\ZZ_2$ veamos que tambien podemos representar el 1, que 
basta verlo para las formas módulo 8, y con eso repetimos el 
razonamiento anterior. Probando un poquito vemos que $5x^2
+3y^2$ es congruente a 1 en el vector $(1,2)$.

Es facil ver que las formas no son equivalentes sobre $\ZZ$
porque la primera puede representar primitivamente el $5$ o el 
$11$ con los vectores $(1,0)$ o $(0,1)$ mientras que la segunda
no pues $x^2 + 55y^2 < 55 \iff y=0$ pero en ese caso solo 
podemos representar cuadrados. Tambien porque son formas
reducidas del mismo discriminante y definidas positivas, pero 
no iguales.

\subsection*{Guía 3, ejercicio 6, item b}

Las matrices de nuestras formas son:

\begin{multicols}{2}
$$\phi_1 = \lp
\begin{matrix}
    5&8&1 \\
    8&13&1 \\
    1&1&11 \\
\end{matrix}\rp$$

$$\phi_2 = \lp
\begin{matrix}
    1&1&-2 \\
    1&16&7 \\
    -2&7&10 \\
\end{matrix}\rp$$
\end{multicols}

Diagonalicemos las formas cuadráticas (las operaciones 
marcadas se realizan en filas y en columnas tambien):

$$
\lp\begin{matrix}
    5&8&1 \\
    8&13&1 \\
    1&1&11 \\
\end{matrix}\rp
\xrightarrow[f_3 \to f_3-\frac{1}{5}f_1]{f_2 \to f_2-
\frac{8}{5}f_1}
\lp\begin{matrix}
    5&0&0 \\
    0&\frac{1}{5}&-\frac{3}{5} \\
    0&-\frac{3}{5}&\frac{54}{5} \\
\end{matrix}\rp
\xrightarrow{f_3 \to f_3+3f_2}
\lp\begin{matrix}
    5&0&0 \\
    0&\frac{1}{5}&0 \\
    0&0&9 \\
\end{matrix}\rp
$$

$$
\lp\begin{matrix}
    1&1&-2 \\
    1&16&7 \\
    -2&7&10 \\
\end{matrix}\rp
\xrightarrow[f_3 \to f_3+2f_1]{f_2 \to f_2-f_1}
\lp\begin{matrix}
    1&0&0 \\
    0&15&9 \\
    0&9&6 \\
\end{matrix}\rp
\xrightarrow{f_3 \to f_3-\frac{3}{5}f_2}
\lp\begin{matrix}
    1&0&0 \\
    0&15&0 \\
    0&0&\frac{3}{5} \\
\end{matrix}\rp
$$

Y entonces tenemos $q_1 \sim \lc 5,\frac{1}{5},9\rc \sim
\lc 5,5,9\rc$ y $q_2 \sim \lc 1,15,\frac{3}{5}\rc\sim 
\lc 1,15,15\rc$ donde las segundas equivalencias solo valen
si 5 es unidad y podemos usarlo en la variable, es decir
para todo $\ZZ_p$ con $p\neq 5$. Notemos que los coeficientes 
solo tienen multiplos de 3 y 5 asi que para cualquier $p$ que
no sea 2, 3, 5 ni $\infty$ las formas van a ser equivalentes 
en $\ZZ_p$. Tambien descartemos rapidamente $\infty$ pues 
ambas formas son definidas positivas.

Primero veamos $\ZZ_5$, donde las formas asi como las 
diagonalizamos no estan definidas pero podemos ver la 
equivalencia entre $5q_1$ y $5q_2$ (cancelamos 
denominadores):
$$5q_1 \sim \lc 25,1,45\rc\sim\lc 1,1,5\rc$$
$$5q_2 \sim \lc 5,75,3\rc \sim\lc 5,3,3\rc$$

Entonces resta ver si $\lc 1,1\rc\sim_{\ZZ_5}\lc 3,3\rc$. 
Como $3\cdot 1^2+3\cdot 1^2=6\equiv 1 \mod 5$ entonces la 
segunda puede representar el 1, y como su determinante es
un cuadrado es equivalente a $\lc 1,1\rc$ como queriamos ver.

Segundo $\ZZ_3$ donde podemos comparar $\lc 5,5,9\rc$ con 
$\lc 1,15,15\rc$. Como los bloques de cada valuación son 
distintos entonces las formas canonicas no van a ser iguales
y podemos descartar directamente la equivalencia incluso en
$\QQ_3$.

Finalmente nos queda $\ZZ_2$ pero tenemos la suerte de poder
evitarlo: como las formas eran equivalentes en $\QQ_p$ para
todo $p\neq 2,3$ y no lo eran en $\QQ_3$ entonces no pueden 
serlo en $\QQ_2$ porque (tienen el mismo determinante en 
$\QQ$ y) el producto de todos los símbolos de Hasse es 1, y 
como $C_3(q_1)\neq C_3(q_2)$ entonces necesariamente 
$C_2(q_1)\neq C_2(q_2)$ y no son equivalentes sobre $\QQ_2$
(mucho menos sobre $\ZZ_2$). 


\subsection*{Guía 3, ejercicio 8} 
Las formas son $q_1=3x^2+2xy+23y^2\sim 3x^2+\frac{68}{3}y^2$ 
y $q_2=7x^2+6xy+11y^2\sim7x^2+\frac{68}{7}y^2$. Son 
equivalentes en $\ZZ_p$ para todo $p\neq2,17$ por ser el 
determinante una unidad (ambas son $\ZZ_p$ equivalentes 
a $x^2+68y^2$ porque siempre podemos poner el determinante
como el ultimo coeficiente y todo el resto unos). Veamos 
esos primos aparte:

$\ZZ_2$: Existe una solución de $q_1(x,y)=7$ para hacer 
el primer cambio de coordenadas, porque mirando módulo 8 
tenemos $3x^2+2xy-y^2 \equiv -1$ que con (0,1) nos da y 
armamos la solución con Hensel. Luego
tenemos $q_1 \sim 7x'^2 + ax'y' + by'^2$, y sabemos que el 
coeficiente $a$ lo podemos cambiar transformando por 
$\minimat{1}{c}{0}{1}$ sumando $7c$ a su variable, pero como 7 
es unidad entonces podemos encontrar solución para $c = 
\frac{6-a}{7}$ en $\ZZ_2$ que nos dará $q_1\sim 7x'^2 + 6x'y' + 
b'y'^2$ y sabemos que el determinante tiene que ser el mismo asi 
que vamos a poder encontrar $b'= 11$ y listo.

$\ZZ_{17}$: Vamos con otra estrategia, usando las formas 
diagonalizadas. Como $3$ y $7$ son unidades, entonces 
$3x^2+\frac{68}{3}y^2\sim 3x^2+3\cdot68y^2$ y lo mismo con
la otra. Tomando la descomposición canonica, como
necesitamos ver que cada componente de distinta valuación 
$17$-adica es equivalente, nos queda unicamente ver que 
$\lc3\rc \sim\lc7\rc$. Veamos que $\leg{3}{17}=\leg{7}{17}$:

$$\leg{3}{17} = \leg{17}{3} = \leg{2}{3} =-1$$
$$\leg{7}{17} = \leg{17}{7} = \leg{3}{7} = (-1)\leg{7}{3} 
= -1$$

Las formas no son $\ZZ$-equivalentes por ejemplo porque la 
primera representa al 3 y el minimo valor que representa la 
segunda es 7 (es definida positiva, esta reducida y es 
primitiva).

Ahora consideremos las formas $q_1=z^2-3x^2-2xy-23y^2$ y 
$q_2=z^2-7x^2-6xy-11y^2$ y probemos que son propiamente 
equivalentes encontrando la transformación entre una y la 
otra.

Es facil ver que $q_2(1,1,z)=z^2-24$ entonces poniendo la 
variable $z=5$ tenemos una representación del 1 no canonica.
Entonces si transformamos la matriz de $q_2$ tendremos un 
uno en la primera casilla, y con eso podemos buscar la 
transformación a $q_1$ con matrices en $\ZZ$:

$$\lp\begin{matrix}
    1&1&5 \\
    0&1&0 \\
    0&0&1 \\
\end{matrix}\rp
\lp\begin{matrix}
    -7&-3&0 \\
    -3&-11&0 \\
    0&0&1 \\
\end{matrix}\rp
\lp\begin{matrix}
    1&0&0 \\
    1&1&0 \\
    5&0&1 \\
\end{matrix}\rp = 
\lp\begin{matrix}
    1&-14&5 \\
    -14&-11&0 \\
    5&0&1 \\
\end{matrix}\rp$$

Para completar la matriz, en la segunda columna podemos 
poner el vector $(2,0,1)$ y obtenemos un 0 al medio arriba, 
y para completar el tercer cero veamos:

$$\lp\begin{matrix}
    1&1&5 \\
    1&0&2 \\
    a&b&c \\
\end{matrix}\rp
\lp\begin{matrix}
    -7&-3&0 \\
    -3&-11&0 \\
    0&0&1 \\
\end{matrix}\rp
\lp\begin{matrix}
    1&1&a \\
    1&0&b \\
    5&2&c \\
\end{matrix}\rp = 
\lp\begin{matrix}
    1&0&C\\
    0&-3&* \\
    C&*&* \\
\end{matrix}\rp$$

donde en $C$ nos queda $-10a-14b+5c$, pero tambien tenemos 
que asegurarnos que el determinante de la matriz, que nos 
queda $2a+3b-c$, sea 1. Tenemos que resolver en enteros:

$$\left\{
\begin{matrix}
    -10a-14b+5c &=&0 \\
    2a+3b-c &=&1 \\
\end{matrix}\right.$$

que sale con $a=0$, $b=5$ y $c=14$. Eso nos dice que en la 
base que encontramos ya logramos una equivalencia entera 
propia entre $q_2$ y la forma $x^2-3y^2+26yz-79z^2$.

Ahora solo nos queda encontrar una equivalencia entera
entre el bloque $\minimat{-3}{13}{13}{-79}$ y el $\minimat
{-3}{-1}{-1}{-23}$ que con un poco de suerte ahora va a existir.

$$\lp\begin{matrix}
    a&c \\
    b&d \\
\end{matrix}\rp
\lp\begin{matrix}
    -3&13 \\
    13&-79 \\
\end{matrix}\rp
\lp\begin{matrix}
    a&b \\
    c&d \\
\end{matrix}\rp = 
\lp\begin{matrix}
    -3&-1\\
    -1&-23 \\
\end{matrix}\rp$$

Para hacerla facil ponemos el (1,0) en la primera columna que
como el $-3$ esta en la primera casilla va bien, y luego 
tenemos que resolver:

$$\left\{
\begin{matrix}
    -3b+13d &=&-1 \\
    -3b^2-79d^2+26bd &=&-23 \\
\end{matrix}\right.$$

$$d=\frac{3b-1}{13}$$

$$-3b^2-\frac{79}{169}(9b^2-6b+1)+2(3b^2-b)=-23$$

$$-\frac{204}{169}b^2-\frac{136}{169}b+\frac{3808}{169}=0$$

$$b=-4, \frac{14}{3}$$

Luego queda $d=-1$ y tenemos la siguiente transformación:

$$\lp\begin{matrix}
    1&0 \\
    -4&-1 \\
\end{matrix}\rp
\lp\begin{matrix}
    -3&13 \\
    13&-79 \\
\end{matrix}\rp
\lp\begin{matrix}
    1&-4 \\
    0&-1 \\
\end{matrix}\rp = 
\lp\begin{matrix}
    -3&-1 \\
    -1&-23 \\
\end{matrix}\rp$$

pero tiene determinante $-1$. Por suerte nos queda un as 
bajo la manga y es que en la matriz habríamos puesto un 1
que correspondía a fijar la otra coordenada, y podemos poner
un $-1$ pues al cuadrado eso no nos cambia. Finalmente:
$$\lp\begin{matrix}
    0&1&0 \\
    0&-4&-1 \\
    -1&0&0 \\
\end{matrix}\rp
\lp\begin{matrix}
    1&1&5 \\
    1&0&2 \\
    0&5&14 \\
\end{matrix}\rp
\lp\begin{matrix}
    -7&-3&0 \\
    -3&-11&0 \\
    0&0&1 \\
\end{matrix}\rp
\lp\begin{matrix}
    1&1&0 \\
    1&0&5 \\
    5&2&14 \\
\end{matrix}\rp
\lp\begin{matrix}
    0&0&-1 \\
    1&-4&0 \\
    0&-1&0 \\
\end{matrix}\rp = 
\lp\begin{matrix}
    -3&-1&0 \\
    -1&-23&0 \\
    0&0&1 \\
\end{matrix}\rp$$
es un cambio de coordenadas entero de determinante 1 entre
las formas $q_2$ y $q_1$, lo que nos dice que son propiamente
equivalentes. Viendo la matriz final, noto que tal vez podía 
ahorrarme la mitad de las cuentas si calculaba una solución 
un poco mejor.


% \subsection*{Guía 4, ejercicio 1} Tenemos $u$ un elemento de grado 
% par con inverso multiplicativo en el álgebra de Clifford, tal que 
% la conjugación $\rho_u$ preserva $V$. Escribamos $u=v_1\cdot v_2
% \cdots v_k$ y veamos que $k$ es par. NO USE QUE ES DE GRADO PAR XD
% PERO OBVIO TA MAL PQ NO VI EL CASO U=ALGUNOS VI + OTROS VI
% LA SOL ERA NQV, HABIA QUE VER QUE TENIAS ISOMETRIA DE V, LUEGO
% CON ESO LA ESCRIBIAS COMO PROD DE VECTORES Y PROBABAS QUE ESO ERA 
% LO MISMO QUE U VIA VER QUE EL PRODUCTO DE UNO POR EL INVERSO DEL 
% OTRO ERA UN ESCALAR (PQ ERA PAR EN EL CENTRO DEL ALGEBRA WTF)

% Usemos el siguiente lema: si $v\in V$ entonces lo podemos ver en 
% $\cont(V)$ y si tiene inverso entonces $\rho_v|_V=-\tau_v$. La 
% demostración es la siguiente cuenta:
% $$\rho_v(x)=vxv^{-1}=\frac{vxv}{q(v)}=\lp -xv+2\phi(x,v)\rp
% \frac{v}{q(v)} = -x+\frac{2\phi(x,v)}{q(v)}v=-\tau_v(x)$$

% Un elemento $v$ de $V$ tiene inverso visto en $\cont(V)$ si y solo si 
% $q(v)\neq 0$ y entonces podemos dividir por $q(v)$. Nos gustaria 
% asegurarnos de que los $v_i$ que tenemos tienen inverso para usar 
% el lema, y esto lo tenemos porque $u$ tiene inverso. Si $u=v_1v_2
% \cdots v_k$ entonces el inverso de $v_1$ es $v_2\cdots v_ku^{-1}$
% (para ver que es inverso a derecha $v_2\cdots v_ku^{-1}v_1 = 1$
% multiplicando a izquierda por $v_2\cdots v_k$ nos queda obvio). 
% Luego vemos que $v_2\cdots v_k$ tiene inverso usando el de $u$ y el
% de $v_1$, y ese inverso será $u^{-1}v_1$:
% $$u^{-1}v_1 \cdot v_2v_3\cdots v_k= uu^{-1}=1$$
% $$v_1 (v_2\cdots v_k \cdot u^{-1}v_1) = uu^{-1}v_1=v_1 \implies 
% v_2\cdots v_k \cdot u^{-1}v_1=1.$$
% Inductivamente tenemos que cada $v_i$ tiene inverso.

% Entonces sabemos que $\rho_{v_i}$ restringido a $V$ es igual a 
% $-\tau_{v_i}$, luego escribiendo $\rho_u=\rho_{v_1}\circ\rho_{v_2}
% \circ\cdots\circ \rho_{v_k}$ tenemos que, restringido a $V$ tiene 
% que ser una composicion de reflexiones, y como cada una tiene 
% determinante $-1$ entonces deben ser una cantidad par para que la 
% composicion quede trivial (en particular de determinante 1).


% $$q(\rho_u(x)) = (\rho_u(x))^2 = (uxu^{-1})^2 = (uxu^{-1})(uxu^{-1})$$


% \subsection*{Guía 4, ejercicio 2}
% Como sabemos que el álgebra de Clifford no depende de la base de 
% nuestro espacio $V$, entonces antes de arrancar diagonalicemos $q$:
% $x^2+xy+3y^2-2z^2 = (x+\frac{y}{2})^2+\frac{11}{4} y^2-2z^2$. Tambien 
% podemos meter el $\frac{1}{4}$ en la variable asi quedan enteros los 
% coeficientes. Si pensamos en los generadores $v_1$, $v_2$ y $v_3$ 
% sabemos que $v_i\cdot v_j=-v_j\cdot v_i$ si $i\neq j$ (porque 
% diagonalizamos) y que $v^2_1=1$, $v^2_2=11$ y $v^2_3=-2$. Esto es
% suficiente para presentar el álgebra sobre $\QQ$. 

% Ahora para encontrar $\tilde{q}(u)=u\varepsilon(u)$ escribamos la 
% base $B=\{1,v_1,v_2,v_3,v_1v_2,v_1v_3,v_2v_3,$
% $v_1v_2v_3\}$ y veamos como se comportan con la multiplicacion por 
% $\varepsilon(u)$ los generadores. Por ejemplo
% $$v_1v_2v_3\varepsilon(v_1v_2v_3)=v_1v_2v_3v_3v_2v_1=
% q(v_3)v_1v_2v_2v_1=q(v_3)q(v_2)q(v_1)$$
% Con eso tenemos los valores de la diagonal de $\tilde{q}$:
% \begin{itemize}
%     \item $\tilde{q}(1)=q(1)=1$
%     \item $\tilde{q}(v_1)=q(v_1)=1$
%     \item $\tilde{q}(v_2)=q(v_2)=11$
%     \item $\tilde{q}(v_3)=q(v_3)=-2$
%     \item $\tilde{q}(v_1v_2)=q(v_1)q(v_2)=11$
%     \item $\tilde{q}(v_1v_3)=q(v_1)q(v_3)=-2$
%     \item $\tilde{q}(v_2v_3)=q(v_2)q(v_3)=-22$
%     \item $\tilde{q}(v_1v_2v_3)=q(v_1)q(v_2)q(v_3)=-22$
% \end{itemize}
% que corresponden a la diagonal. Si nos interesa esta forma cuadratica
% como la norma espinoral entonces esto es suficiente pues solo será 
% evaluada sobre elementos de $O_q$ que no tienen sumas (toda 
% transformación se puede escribir como producto de reflexiones).

% Los productos cruzados de vectores nos quedarán con componentes en el 
% álgebra de dimension mayor a 0, entonces estos son los unicos valores 
% que pertenecen a $\QQ$.

% \subsection*{Guía 4, ejercicio 4}

% Como $(V,q)$ es regular entonces la podemos diagonalizar, siendo 
% $B_V:= \{ v_1,v_2,v_3 \}$ una base ortogonal, con $q(v_i)=a_i$ tales 
% que $a_1a_2a_3\equiv d (k^\times)^2$; y una base del álgebra
% será $B_\cont\{ 1,v_1,v_2,v_3,v_1v_2,v_1v_3,v_2v_3,v_1v_2v_3 \}$.

% Observamos en la teórica que el centro del algebra tiene siempre a 
% los escalares (obvio) y a la componente de mayor grado del algebra si
% la dimensión del espacio era impar (3 en nuestro caso), asi que 
% veamos que el elemento $\vvv := v_1v_2v_3$ (que ya sabemos que está 
% en el centro) cumple $\vvv^2 \equiv -d (k^\times)^2$:
% $$\vvv^2=(v_1v_2v_3)(v_1v_2v_3) = v_1v_2(-v_1v_3)v_2v_3=
% v_1(v_1v_2)(-v_2v_3)v_3 = a_1(-a_2)a_3\equiv -d (k^\times)^2$$

% Ahora para el siguiente punto supongamos que $(V,q)$ es isotropico,
% asi que vamos a cambiar su base a una que haga uso de ello. Sabemos 
% que ahora tiene un plano hiperbolico incluido, y por ser de 
% dimensión 3 y regular entonces $H^\perp=\lc z\rc$ es de dimensión 1.
% Si llamamos $v$ y $w$ a la base que nos da el $H$ entonces en $\cont$
% valen:

% $v^2=w^2=0$
% $vw+wv=2$
% $z^2=a\neq 0$
% $vz=-zv$ 
% $wz=-zw$

% y la base para $\cont_0(V)$ es $\{1,vw,vz,wz\}$, veamos a que matrices
% podemos asignarlos: para que sea un isomorfismo $\phi(1)=I$, luego 
% $(vz)^2=vz(-zv)=0\cdot a=0$ y de la misma forma $(wz)^2=0$, tambien
% $(vw)^2=vw(2-wv)=2vw$ y $vw\cdot wz=0vz=0$. 

% El iso saale pero la condicion extra no 








\newpage

\vspace{1em}
\noindent\textbf{Código 1.} Potencia grande de 2 menos 1 módulo 163
\label{codigo1}

\lstset{
    language=Python,         % Cambiá si usás otro lenguaje
    basicstyle=\ttfamily\small,
    frame=single,
    breaklines=true
}

\begin{lstlisting}
import numpy as np
#esto lo tendria que hacer con bits y no listas pero la idea esta
def resta_binaria(cant_de_unos: int, tu_num:int):
    bin_str = bin(tu_num)[2:]  # pasa a binario y elimina el prefijo '0b'
    bits = [int(b) for b in bin_str]
    el_chico = np.array(bits, dtype=int)

    el_grandote = np.ones(cant_de_unos, dtype=int)
    while(len(el_grandote) > len(el_chico)):  #no para exactamente donde deberia pero era mas simple
        el_grandote = grande_chico(el_grandote, el_chico)
        el_grandote = recortar_ceros_iniciales(el_grandote)
        
    valor_final = int(''.join(str(b) for b in el_grandote), 2)   #esta cosa fea pasa de lista de bits a representacion decimal
    if(valor_final >= tu_num):   #aca por si me falto una iteracion
        valor_final -= tu_num
        
    return valor_final

def grande_chico(grande: np.ndarray , chico: np.ndarray): #osea grande-chico
    #primero vemos si entra bien el num:
    ind = 0
    while(ind < len(chico) and grande[ind] == chico[ind]):  #el indice en que difieren
        ind +=1
    if(ind < len(chico) and grande[ind] < chico[ind]):  #depende para donde difieran
        chico = np.insert(chico, 0, 0)
    #ahora los restamos pq seguro entra bien (recorremos la resta como corresponde arrancando del menor)
    for i in range(len(chico) - 1, -1, -1):  #resta coord a coord
        if(grande[i] < chico[i]):  #observar que esto es unicamente si son 0 y 1 respectivamente
            grande[i] = 1
            j = 1
            while(grande[i-j] == 0):   #propagamos la resta
                grande[i-j] = 1
                j+=1
            grande[i-j] = 0
        else:   #restamos normalmente
            grande[i] -= chico[i]
    return grande

def recortar_ceros_iniciales(arr):  #para ir achicando la lista
    for i, val in enumerate(arr):
        if val != 0:
            return arr[i:]
    return np.array([], dtype=arr.dtype)

resultado = resta_binaria(131, 163)
ya_que_estamos = resta_binaria(9689, 163)

print(resultado)
print(ya_que_estamos)   
\end{lstlisting}


\vspace{1em}
\noindent\textbf{Código 2.} Calculando el resultado
\label{codigo2}

\lstset{
    language=Python,
    basicstyle=\ttfamily\small,
    frame=single,
    breaklines=true
}

\begin{lstlisting}
K163 = QuadraticField(-163, 'a')
OK163 = K163.ring_of_integers()
a = K163.gen()
r = power_mod(163, 2^9687, 2^9689 - 1)
p = 2^9689 - 1

p163 = OK163(p)
x = OK163(r + a)
resultado = p163.gcd(x)
print(resultado)
\end{lstlisting}

\vspace{1em}
\noindent\textbf{Código 3.} Potencia numerica
\label{codigo3}

\lstset{
    language=Python,
    basicstyle=\ttfamily\small,
    frame=single,
    breaklines=true
}

\begin{lstlisting}
def busco_potencia_numerica(numero: int, iteraciones: int, prec_de_R: int):
    R = RealField(prec_de_R)
    raiz = 1
    while(raiz < iteraciones + 1):
        res = numero^(1/raiz)
        print(raiz, R(res+res^(-1)))
        raiz = raiz+1

busco_potencia_numerica(1545688332149051, 20, 200)
\end{lstlisting}

\newpage

\begin{equation*} \label{solucion2}
\begin{array}{l}
\noindent\textbf{Solución ejercicio 11:} \\
83250108517244754631958379704346169736009070717927333661152717076768610043 \\
54332925305504156237999402648944860482471511367497546427114931066938614310 \\
63601894373330994412046249817232863797706783597434079965470573504747236845 \\
48312506559576111972357424519311070561701236243581147439658472127676965209 \\
62764442506744894217903407066796961795602023959971203575468909439143651204 \\
40429078236276891066118990402208435946171211041961862906954844572844984140 \\
28337213687092229363834423692456794657294141899857796332199823886688976978 \\
16441168460334899931350398896569697234824356108695845964274413019895614057 \\
13960149203756747970116076469933847463897211681623626531637571075739794962 \\
21021125453522437793256114019381530238860844355639257051174510822589137213 \\
30852316723444239079026615463065655121975250626918950835541038100333373202 \\
79013794406125065502197916976866344914241262892787759190457600979783016878 \\
56535114583175392343955351100932141247542145127279674285526410302492365249 \\
90955880171046459284068927574801405665404564173300943268595999682975727177 \\
36653086087129420198777022717344958224340755178261524941901846235896332568 \\
97288705319201088812517312916758457637266802053149224201063207326903532878 \\
38346185479703600416340317489511087198940097563130815098399330788611636469 \\
63298305365750990624650207374682514435284350951496536071463217673230269214 \\
13793191660134855343245232296119359841157283028354741388287972657089945146 \\
2148923487030255687083976986475444970862685049813678 ^2 + 163 \cdot 15838805981600 \\
37155614749348213488327702363871007708307190536733624647278148713436694255 \\
40790943324335246687351795477862279550861529893558370741086932980801317328 \\
28745913352764649844379126748772916485743891357523834352930403484561330483 \\
12467072250126761095922129197790802311128866096926886647080689896969147985 \\
55812060435793524996158114987223491941271438983414054545172535990331173921 \\
38555307564984038683535994801790613869155837861634194800595359621251744648 \\
63151024664889600570523314147491562120554151297846518628266437415185613168 \\
71522826230841839237452002018714065095825581010558666345089395823370484776 \\
45038255347043213310172937419173120975419021107830448590722436867041807717 \\
95294739128968948257099168361452814288531796443802478296604736721011084475 \\
17819168565177377298595914165632297886581611428376480651653912934395127513 \\
70779058987437725724251135390999892314196160646143974980185584932741273525 \\
07712236815243682161262983123289211100355372396487832066767704118898545657 \\
51359463167950999148856978044543678441563674076983704364818722325559487088 \\
01081522876821710018052852199045447550701318228100146254397141919977856640 \\
58078342802290716714257936224132201215345891991491645275420107817726645086 \\
61234781446680846883658486011912867247810843465460101608858945128763246836 \\
25036330971567142901071669527089130748189333085609398511858013966946006288 \\
28188856969277887360171723464015604266504792042630184125012402353281338641 \\
39948140111815854430969832618590706173 ^2 = 2^{9698}-1 
\end{array}
\end{equation*}

\begin{equation*} \label{solucion25}
\begin{array}{l}
\noindent\textbf{Solución ejercicio 25:} \\
101041943787986597181019507863069141934440128513289455403365326172557189585 \\
676715679665385743712020813459164774126677555944022198204046811476842090471 \\
16292994007696886678349396097^2- 17\cdot97^2 \cdot25264196898459124852442842075115423 \\
167539340934798316032474205305179748194350162065076985696972535123781526196 \\
828871485645802992181329240509507734477735657580023979143458832144^2 = 1
\end{array}
\end{equation*}

\end{document}