\documentclass[12pt]{amsart}

%Fuente
\usepackage[utf8]{inputenc}
\usepackage[spanish]{babel}
\usepackage[T1]{fontenc}
\usepackage{lmodern}
\usepackage{epigraph}
\usepackage{lipsum}
\usepackage[a4paper, margin=2.5cm]{geometry}
\usepackage{multicol}
\usepackage{amsmath}
\usepackage{amssymb}
\usepackage[shortlabels]{enumitem}
\usepackage{listings}
\DeclareMathOperator{\mcd}{mcd}

\renewcommand{\refname}{Bibliografía}

%diagramas
\usepackage{tikz-cd}

%enumerar páginas y encabezados
\pagestyle{headings}

%Comandos para escribir C más rápido
\usepackage{mathrsfs}
\newcommand{\CC}{\mathbb{C}}
\newcommand{\NN}{\mathbb{N}}
\newcommand{\QQ}{\mathbb{Q}}
\newcommand{\RR}{\mathbb{R}}
\newcommand{\ZZ}{\mathbb{Z}}
\newcommand{\FF}{\mathbb{F}}
\newcommand{\cont}{\mathcal{C}}

\newcommand{\pp}{\mathfrak{p}}

\newcommand{\aaa}{\mathfrak{a}}
\newcommand{\bbb}{\mathfrak{b}}

\newcommand{\OO}{\mathcal{O}}

\newcommand{\leg}[2]{\left( \frac{#1}{#2} \right)}
\newcommand{\minimat}[4]{\left(\begin{smallmatrix} #1 & #2 \\ #3 & #4 
\end{smallmatrix}\right)}
\newcommand{\lp}{\left(}
\newcommand{\rp}{\right)}
\newcommand{\lb}{\left|}
\newcommand{\rb}{\right|}
\newcommand{\lc}{\left<}
\newcommand{\rc}{\right>}
\newcommand{\lco}{\left[}
\newcommand{\rco}{\right]}

%dobles implicaciones
\usepackage{csquotes}

%Entornos con estilo 'tipo teorema'
\theoremstyle{plain}
\newtheorem{teo}[equation]{Teorema}
\newtheorem{ej}[equation]{Ejercicio}
\newtheorem{prop}[equation]{Proposición}
\newtheorem{lem}[equation]{Lema}
\newtheorem{coro}[equation]{Corolario}
\newtheorem{defi}[equation]{Definición}
\newtheorem{obs}[equation]{Observación}
\newtheorem*{nota*}{Notación}
\renewcommand*{\proofname}{Demostración}



%Hipervínculos cliqueables, en color azul
\usepackage[colorlinks=true,allcolors=blue]{hyperref}

\title{Entrega de ejercicios de teoría de números}

\author{Pedro Nicolas Peña}

\begin{document}

% \begin{abstract}
%     Buee quien era
% \end{abstract}

\maketitle

\section*{Guía 1}

\subsection*{Ejercicio 4}Implementar el siguiente algoritmo en SAGE: Para encontrar un $\alpha$ perteneciente al anillo de enteros de un cuerpo de números $K$ de grado $n$, dado $R=\lc \alpha_1,\cdots,\alpha_n\rc\subseteq\OO_K$ un $\ZZ$-módulo de rango $n$ que no es $\OO_K$. Encontrar el primo $p$ tal que $p^2|\Delta_K$ y los enteros $c_1,c_2,\cdots,c_n$ entre $0$ y $p$ tales que 
$$\frac{1}{p}\sum_{i=1}^n c_i\alpha_i$$ 
pertenece a $\OO_K$.

\subsection*{Ejercicio 15}
Sea $K=\QQ(\zeta)$ con $\zeta = e^{2\pi i /5}$
\begin{enumerate}[a)]
    \item Probar $\zeta^2+\zeta^3 \in \OO_K^\times$.
    \item Deducir que $\OO_K^\times$ es infinito.
\end{enumerate}

\section*{Guía 2}

\subsection*{Ejercicio 4} $\textit{Teorema de aproximacion débil}$

Sea $\OO$ un dominio de Dedekind. Sean $\pp_1, \cdots ,\pp_n \unlhd \OO$
primos no nulos distintos dos a dos, y para cada $1\leq i \leq n$ sea 
$e_i\in \NN_0$.

Probar que existe $\alpha \in \OO$ tal que 
$$val_{\pp_i}(\alpha)=e_i, \ 1\leq i \leq n.$$
$\textit{Sugerencia: usar el teorema chino del resto.}$

\subsection*{Ejercicio 5} Sea $\OO$ un dominio Dedekind que tiene un 
número finito de ideales primos.

Probar que $\OO$ es un DIP.

\section*{Guía 3}

\subsection*{Ejercicio 6} Sean $m,n$ enteros libres de cuadrados con 
$m\neq n$. Sea $K=\QQ(\sqrt{n}, \sqrt{m})$. \textit{Observar que 
$K/\QQ$ tiene exactamente tres subextensiones cuadráticas}.

Sea $p$ un primo.

\begin{enumerate}[a)]
    \item Dar ejemplos en los que:
    \begin{enumerate}
        \item[\textup{i.}] $p$ se parta completamente en $K$.
        \item[\textup{ii.}] $p\OO_K=P_1^2P_2^2$, con $P_1,P_2\unlhd
        \OO_K$ primos.
    \end{enumerate}

    \item Probar que no puede suceder que:
    \begin{enumerate}
        \item[\textup{i.}] $p$ sea inerte en $K$.
        \item[\textup{ii.}] $p$ sea impar y totalmente ramificado en
        $K$.
    \end{enumerate}
\end{enumerate}

\subsection*{Ejercicio 10} Sea $f \in \ZZ[X]$.

\begin{enumerate}[a)]
    \item Probar que existen infinitos primos $p$ tales que $\bar{f}\in
    \FF_p[X]$ tiene al menos una raíz en $\FF_p$.

    \textit{Sugerencia: probarlo primero en el caso en que $f(0) = 1$, 
    notando que en tal caso para todo $m \geq 2$ se tiene que 
    $f(km)\equiv 1 (mod\ m)$ para todo $k\in \ZZ$}.
    
    \item \begin{enumerate}
        \item[\textup{i)}] Probar que si $K/\QQ$ es un cuerpo de números, existen 
        infinitos primos $P$ de $\OO_K$ tales que $f(P|P\cap\ZZ)=1$.

        \item[\textup{ii)}] Probar que si $K\subseteq L$ son cuerpos de números, 
        existen infinitos primos $P$ de $\OO_K$ tales que para todo 
        primo $Q$ de $\OO_L$ con $Q\cap K = P$ se tiene que $f(Q|P)=1$.

        \textit{Sugerencia: probarlo primero para el caso en que $L/K$
        es Galois}.
    \end{enumerate}
    
    \item Probar que existen infinitos primos $p$ tal que $\bar{f}\in
    \FF_p[X]$ tiene $todas$ sus raices en $\FF_p$.
\end{enumerate}

\section*{Guía 4}

\subsection*{Ejercicio 4} 

\subsection*{Ejercicio 11} 

\newpage

\section*{Soluciones}

\subsection*{Guía 1, ejercicio 4}
Un poco hablemos de la solución del ejercicio que presenta el resultado
teorico para explicar el cálculo. Primeramente sabemos que como 
$R \subsetneq \OO_K$ es un orden (por definicion subanillo de $\OO_K$
de rango máximo). Entonces cumple que $\Delta_R = \Delta_K [\OO_K:R]$


\subsection*{Guía 1, ejercicio 15}
Los conjugados de $\zeta$ son sus potencias, entonces calcular la norma
de $\zeta^2+\zeta^3$ va a ser multiplicar lo siguiente:
$$(\zeta^2+\zeta^3)(\zeta^4+\zeta^6)(\zeta^6+\zeta^9)(\zeta^8+\zeta^{12})
= (\zeta^2+\zeta^3)(\zeta^4+\zeta^1)(\zeta^1+\zeta^4)(\zeta^3+\zeta^2)=$$
$$=\lp(\zeta^2+\zeta^3)(\zeta^4+\zeta^1)\rp^2=
(\zeta^6+\zeta^3+\zeta^7+\zeta^4)^2=(\zeta^1+\zeta^3+\zeta^2+\zeta^4)^2=
1$$
porque $\zeta$ es raiz de $X^4+X^3+X^2+X+1 = \frac{X^5-1}{X-1}$.

Como ahora sabemos que $\zeta^2+\zeta^3$ tiene norma 1 y es suma de 
enteros algebraicos pues (de vuelta) $\zeta$ es raiz de un polinomio con
coeficientes enteros, entonces pertenece a las unidades del anillo de 
enteros. Pero tambien como vemos que su valor no es ni 0 ni 1 ni -1 y 
es real (geometricamente porque $\bar{\zeta^2}=\zeta^{-2} = 
\zeta^3$ o sino porque 
$(\zeta^2+\zeta^3)^2=\zeta^4+\zeta^1+2$ entonces $\zeta^2+\zeta^3$ 
será raiz de $X^2+X-1$ que es el famoso $\bar{\varphi}$ que cae en 
$\QQ[\sqrt{5}]\subseteq \RR$) entonces todas sus potencias serán 
distintas y tendrán la misma norma, dando lugar a infinitos 
elementos en $\OO_K^\times$.


\subsection*{Guía 2, ejercicio 4}

El teorema chino del resto dice que si $\aaa = \aaa_1 \cdots \aaa_r$ 
son ideales coprimos (en 
nuestro caso seran $\pp_i^{e_i+1}$ primos distintos) entonces 
$$\OO / \aaa \cong \bigoplus_{i=1}^r \OO / \aaa_i$$
y en la demostracion se prueba que el morfismo $\varphi:\OO\to
\oplus_{i=1}^r \OO / \aaa_i$ dado por $\lp\varphi(\alpha)\rp_i=\alpha\mod 
\aaa_i$ tiene nucleo $\aaa$ y es sobreyectivo.

Para resolver el problema es suficiente probar que existe un elemento 
$\alpha \in \OO$ tal que $\alpha \equiv0\mod \pp_i^{e_i}$ pero 
$\alpha \not\equiv 0 \mod \pp_i^{e_i+1}$ 
simultaneamene para todo $i$ y la gran ventaja que da el teorema es que 
basta verlo en cada coordenada y luego tomar preimagen. Ahora como 
$\pp_i^{e_i+1} \subsetneq \pp_i^{e_i}$ (son distintos porque estan 
factorizados
como producto de primos y sus factorizaciones son distintas), entonces 
existen elementos $\alpha_i \in \pp_i^{e_i} - \pp_i^{e_i+1}$ para todo $i$.
Luego tomamos $\alpha \in \bigoplus_{i=1}^r \OO / \pp_i^{e_i+1}$ tal que
$(\alpha)_i \mod \bar\alpha_i\mod \pp_i^{e_i+1}$ y el elemento $\varphi^{-1}
(\alpha)$ cumplirá lo que queremos (cualquier representante en $\OO$).


\subsection*{Guía 2, ejercicio 5}

Queremos ver que dado un ideal $\aaa = \pp_1^{e_1}\cdots\pp_n^{e_n}$ 
esta generado por un solo elemento $\alpha$. Vamos a usar la 
definicion de dominio de Dedekind que nos dice que todo ideal se 
factroriza de forma unica como producto de ideales primos.
Como los ideales primos son finitos vamos a escribirlos como $\pp_1$
hasta $\pp_n$ de modo que en las factorizaciones aparezcan todos (los
exponentes pueden ser $0$).

Ahora tomo $\alpha$ como el que nos da el ejercicio anterior, con las 
valuaciones iguales a la factorizacion de nuestro ideal. Por tener 
factorizacion unica en ideales primos entonces $(\alpha)$ se escribirá
de una unica forma como $\pp_1^{f_1}\cdots\pp_n^{f_n}$.

Por ser $\OO$ un Dedekind vale que el producto de ideales primos es la 
interseccion, y como $\alpha \in \pp_i^{e_i}$ (por definicion de valuacion)
para todo $i$ entonces $\alpha \in \aaa$ con lo que $(\alpha)\subseteq\aaa$.
Es decir $f_i\geq e_i$. Pero como $\alpha \notin \pp_i^{e_i+1}$ entonces 
debe ser que $f_i=e_i$ para todo $i$.


\subsection*{Guía 3, ejercicio 10}

Sea $f\in \ZZ[X]$. Si $f(0)=1$ entonces, como el coeficiente 
independiente es 1, al evaluar el polinomio en multiplos de $m$ y 
tomar módulo $m$ solo nos quedará el coeficiente independiente. Ahora,
visto en $\ZZ$ el polinomio no puede valer infinitas veces $1$ pues 
sería constante (si $n$ es el grado del polinomio entonces en $f(m), 
f(2m), \cdots, f(nm), f((n+1)m)$ debe haber alguno distinto a 1 
porque sinó $f-1$ tendria el mismo grado y mas de $n$ raices). 

Supongamos que $f$ tiene una raiz solamente en finitos primos $p_1,
\cdots, p_n$, tomemos $m=\prod p_i$ y veamos $f(km) = jm+1$ donde en
algun momento $j$ será distinto de 0, entonces $jm+1$ no va a ser 
divisible por ninguno de los $p_i$, es decir que en su factorizacion 
en primos va a aparecer algun $p_{n+1}$ nuevo, y en $\FF_{p_{n+1}}$
el polinomio tendrá una raíz.

Ahora si $f(0) = n$ no nos podemos asegurar que $f(km) = jm+n$ no 
sea un multiplo de los primos de $m$ pues puede darse que $n$ sea un
multiplo de $m$, pero esto se soluciona sacando de $m$ los primos que 
dividan a $n$ (que deberiamos incluirlos porque ahí obviamente vale 
$f(0)\equiv 0$). En resumen: defino $m$ como el producto de los primos
donde $\bar{f}$ tiene una raíz (asumiendo que son finitos) salvo 
aquellos que dividan a el coeficiente independiente, luego como 
$f(km) \equiv n \mod m$ en algun momento valdrá en $\ZZ$ que 
$f(km) = jm+n$ y ese valor tendrá factores primos distintos a los de 
$m$ (si $p|jm+n$ y $p|m$ entonces tambien $p|n$ pero esos los 
excluimos), llegando así a un absurdo pues $\bar{f}$ tendría una raíz 
en ese $\FF_p$.

Ahora pasemos al lenguaje de teoría de números. Dado $K$ un cuerpo de 
números, sabemos que podemos escribir $K=\QQ(\alpha)$ y llamemos $f$ al
polinomio minimal de $\alpha$. Tenemos la
conexión de que si nuestro polinomio $f$ se factoriza como 
producto de irreducibles módulo $p$ como $f_1^{e_1}\cdots f_r^{e_r}$ 
entonces los ideales $\pp_i = p\OO + f_i(\alpha)\OO$ son primos distintos
sobre $p$ con grado de inercia $gr(f_i)$ y $p\OO$ se factoriza como
$\pp_1^{e_1}\cdots \pp_r^{e_r}$. En particular, si $f$ tiene una raíz 
en $\FF_p$ entonces su fatorizacion módulo $p$ tendrá un componente 
lineal y eso nos dirá que el grado de inercia es 1. Por el punto $a$ esto
sucede para infinitos primos.

Si tenemos $\QQ\subseteq K\subseteq L$ entonces ya vimos que hay primos
$Q \unlhd \OO_L$ sobre un $p$ tales que $f(Q | p)=1$ y como el grado 
de inercia es multiplicativo en torres entonces 
tenemos que $f(Q | P)f(P|p)=1$ y son ambos 1. Como en las extensiones de 
Galois todos los primos sobre un determinado $P$ tienen el mismo grado 
de ramificacion y de inercia, entonces valdrá para el resto de los $Q|P$.
Luego si $K\subseteq L$ no es Galois podemos hacer el mismo razonamiento
partiendo de una $L\subseteq L'$ Galois sobre $K$ y $f(Q' | Q)f(Q | P)
f(P|p)=1$ nos dice que son todos $1$ para todo $Q'|P$ primo de $\OO_{L'}$.

Al usar la teoria ramificacion de primos en extensiones junto con Galois probamos que hay infinitos primos que son totalmente ramificados y, volviendo al lenguaje de nuestro polinomio $f$, esto nos dice que $\bar{f}\in \FF_p[X]$ se factoriza como producto de lineales, es decir tiene todas sus raices en el cuerpo.
















\end{document}