\documentclass[12pt]{amsart}

%Fuente
\usepackage[utf8]{inputenc}
\usepackage[spanish]{babel}
\usepackage[T1]{fontenc}
\usepackage{lmodern}
\usepackage{epigraph}
\usepackage{lipsum}
\usepackage[a4paper, margin=2.5cm]{geometry}
\usepackage{multicol}
\usepackage{amsmath}
\usepackage{enumerate}
\usepackage{amssymb}
\usepackage[shortlabels]{enumitem}
\usepackage{listings}
\DeclareMathOperator{\mcd}{mcd}

\renewcommand{\refname}{Bibliografía}

%diagramas
\usepackage{tikz-cd}

%enumerar páginas y encabezados
\pagestyle{headings}

%Comandos para escribir C más rápido
\usepackage{mathrsfs}
\newcommand{\CC}{\mathbb{C}}
\newcommand{\NN}{\mathbb{N}}
\newcommand{\QQ}{\mathbb{Q}}
\newcommand{\RR}{\mathbb{R}}
\newcommand{\ZZ}{\mathbb{Z}}
\newcommand{\FF}{\mathbb{F}}
\newcommand{\cont}{\mathcal{C}}

\newcommand{\aaa}{\mathbf{a}}
\newcommand{\bbb}{\mathbf{b}}

\newcommand{\OO}{\mathcal{O}}

\newcommand{\leg}[2]{\left( \frac{#1}{#2} \right)}
\newcommand{\minimat}[4]{\left(\begin{smallmatrix} #1 & #2 \\ #3 & #4 
\end{smallmatrix}\right)}
\newcommand{\lp}{\left(}
\newcommand{\rp}{\right)}
\newcommand{\lb}{\left|}
\newcommand{\rb}{\right|}
\newcommand{\lc}{\left<}
\newcommand{\rc}{\right>}
\newcommand{\lco}{\left[}
\newcommand{\rco}{\right]}

%dobles implicaciones
\usepackage{csquotes}

%Entornos con estilo 'tipo teorema'
\theoremstyle{plain}
\newtheorem{teo}[equation]{Teorema}
\newtheorem{ej}[equation]{Ejercicio}
\newtheorem{prop}[equation]{Proposición}
\newtheorem{lem}[equation]{Lema}
\newtheorem{coro}[equation]{Corolario}
\newtheorem{defi}[equation]{Definición}
\newtheorem{obs}[equation]{Observación}
\newtheorem*{nota*}{Notación}
\renewcommand*{\proofname}{Demostración}



%Hipervínculos cliqueables, en color azul
\usepackage[colorlinks=true,allcolors=blue]{hyperref}

\title{Entrega de ejercicios de teoría de números}

\author{Pedro Nicolas Peña}

\begin{document}

% \begin{abstract}
%     Buee quien era
% \end{abstract}

\maketitle

\section*{Guía 1}

\subsection*{Ejercicio 4}
Implementar el siguiente algoritmo en SAGE: Para encontrar un $\alpha$
perteneciente al anillo de enteros de un cuerpo de números $K$ de 
grado $n$, dado $R=\lc \alpha_1,\cdots,\alpha_n\rc\subseteq\OO_K$ un 
$\ZZ$-módulo de rango $n$ que no es $\OO_K$. Encontrar el primo $p$ 
tal que $p^2|\Delta_K$ y los enteros $c_1,c_2,\cdots,c_n$ entre $0$ 
y $p$ tales que
$$\frac{1}{p}\sum_{i=1}^n c_i\alpha_i$$
pertenece a $\OO_K$.

% \begin{enumerate}[1.]
% \item Probar que las formas cuadráticas
%     \begin{itemize}
%         \item $g(x,y)=3x^2+2xy+23y^2$
%         \item $h(x,y)=7x^2+6xy+11y^2$
%     \end{itemize}
%     estan en el mismo género pero no son equivalentes.
% \item Probar que las formas ternarias indefinidas $z^2-g(x,y)$ y 
%     $z^2-h(x,y)$ son equivalentes propiamente.

%     \textit{Sugerencia: hallar un vector no canonico de norma 1 para 
%     la segunda y hacer un cambio de variables.}
% \end{enumerate}






\newpage

\section*{Soluciones}

\subsection*{Guía 1, ejercicio 4}
Un poco hablemos de la solución del ejercicio que presenta el resultado
teorico para explicar el cálculo. Primeramente sabemos que como 
$R \subsetneq \OO_K$ es un orden (por definicion subanillo de $\OO_K$
de rango máximo)


\end{document}