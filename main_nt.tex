\documentclass[12pt]{amsart}

%Fuente
\usepackage[utf8]{inputenc}
\usepackage[spanish]{babel}
\usepackage[T1]{fontenc}
\usepackage{lmodern}
\usepackage{epigraph}
\usepackage{lipsum}
\usepackage[a4paper, margin=2.5cm]{geometry}
\usepackage{multicol}
\usepackage{amsmath}
\usepackage{amssymb}
\usepackage[shortlabels]{enumitem}
\usepackage{listings}
\DeclareMathOperator{\mcd}{mcd}

\renewcommand{\refname}{Bibliografía}

%diagramas
\usepackage{tikz-cd}

%enumerar páginas y encabezados
\pagestyle{headings}

%Comandos para escribir C más rápido
\usepackage{mathrsfs}
\newcommand{\CC}{\mathbb{C}}
\newcommand{\NN}{\mathbb{N}}
\newcommand{\QQ}{\mathbb{Q}}
\newcommand{\RR}{\mathbb{R}}
\newcommand{\ZZ}{\mathbb{Z}}
\newcommand{\FF}{\mathbb{F}}
\newcommand{\cont}{\mathcal{C}}

\newcommand{\pp}{\mathfrak{p}}

\newcommand{\aaa}{\mathfrak{a}}
\newcommand{\bbb}{\mathfrak{b}}

\newcommand{\OO}{\mathcal{O}}

\newcommand{\leg}[2]{\left( \frac{#1}{#2} \right)}
\newcommand{\minimat}[4]{\left(\begin{smallmatrix} #1 & #2 \\ #3 & #4 
\end{smallmatrix}\right)}
\newcommand{\lp}{\left(}
\newcommand{\rp}{\right)}
\newcommand{\lb}{\left|}
\newcommand{\rb}{\right|}
\newcommand{\lc}{\left<}
\newcommand{\rc}{\right>}
\newcommand{\lco}{\left[}
\newcommand{\rco}{\right]}

%dobles implicaciones
\usepackage{csquotes}

%Entornos con estilo 'tipo teorema'
\theoremstyle{plain}
\newtheorem{teo}[equation]{Teorema}
\newtheorem{ej}[equation]{Ejercicio}
\newtheorem{prop}[equation]{Proposición}
\newtheorem{lem}[equation]{Lema}
\newtheorem{coro}[equation]{Corolario}
\newtheorem{defi}[equation]{Definición}
\newtheorem{obs}[equation]{Observación}
\newtheorem*{nota*}{Notación}
\renewcommand*{\proofname}{Demostración}



%Hipervínculos cliqueables, en color azul
\usepackage[colorlinks=true,allcolors=blue]{hyperref}

\title{Entrega de ejercicios de teoría de números}

\author{Pedro Nicolas Peña}

\begin{document}

% \begin{abstract}
%     Buee quien era
% \end{abstract}

\maketitle

% con Alt + Z te acomoda las lineas

\section*{Guía 1}

\subsection*{Ejercicio 4}Implementar el siguiente algoritmo en SAGE: Para encontrar un $\alpha$ perteneciente al anillo de enteros de un cuerpo de números $K$ de grado $n$, dado $R=\lc \alpha_1,\cdots,\alpha_n\rc\subseteq\OO_K$ un $\ZZ$-módulo de rango $n$ que no es $\OO_K$. Encontrar el primo $p$ tal que $p^2|\Delta_K$ y los enteros $c_1,c_2,\cdots,c_n$ entre $0$ y $p$ tales que 
$$\frac{1}{p}\sum_{i=1}^n c_i\alpha_i$$ 
pertenece a $\OO_K$.

\subsection*{Ejercicio 15}
Sea $K=\QQ(\zeta)$ con $\zeta = e^{2\pi i /5}$
\begin{enumerate}[a)]
    \item Probar $\zeta^2+\zeta^3 \in \OO_K^\times$.
    \item Deducir que $\OO_K^\times$ es infinito.
\end{enumerate}

\section*{Guía 2}

\subsection*{Ejercicio 4} $\textit{Teorema de aproximacion débil}$

Sea $\OO$ un dominio de Dedekind. Sean $\pp_1, \cdots ,\pp_n \unlhd \OO$
primos no nulos distintos dos a dos, y para cada $1\leq i \leq n$ sea 
$e_i\in \NN_0$.

Probar que existe $\alpha \in \OO$ tal que 
$$val_{\pp_i}(\alpha)=e_i, \ 1\leq i \leq n.$$
$\textit{Sugerencia: usar el teorema chino del resto.}$

\subsection*{Ejercicio 5} Sea $\OO$ un dominio Dedekind que tiene un 
número finito de ideales primos.

Probar que $\OO$ es un DIP.

\section*{Guía 3}

\subsection*{Ejercicio 6} Sean $m,n$ enteros libres de cuadrados con 
$m\neq n$. Sea $K=\QQ(\sqrt{n}, \sqrt{m})$. \textit{Observar que 
$K/\QQ$ tiene exactamente tres subextensiones cuadráticas}.

Sea $p$ un primo.

\begin{enumerate}[a)]
    \item Dar ejemplos en los que:
    \begin{enumerate}
        \item[\textup{i.}] $p$ se parta completamente en $K$.
        \item[\textup{ii.}] $p\OO_K=P_1^2P_2^2$, con $P_1,P_2\unlhd
        \OO_K$ primos.
    \end{enumerate}

    \item Probar que no puede suceder que:
    \begin{enumerate}
        \item[\textup{i.}] $p$ sea inerte en $K$.
        \item[\textup{ii.}] $p$ sea impar y totalmente ramificado en
        $K$.
    \end{enumerate}
\end{enumerate}

\subsection*{Ejercicio 10} Sea $f \in \ZZ[X]$.

\begin{enumerate}[a)]
    \item Probar que existen infinitos primos $p$ tales que $\bar{f}\in
    \FF_p[X]$ tiene al menos una raíz en $\FF_p$.

    \textit{Sugerencia: probarlo primero en el caso en que $f(0) = 1$, 
    notando que en tal caso para todo $m \geq 2$ se tiene que 
    $f(km)\equiv 1 (mod\ m)$ para todo $k\in \ZZ$}.
    
    \item \begin{enumerate}
        \item[\textup{i)}] Probar que si $K/\QQ$ es un cuerpo de números, existen 
        infinitos primos $P$ de $\OO_K$ tales que $f(P|P\cap\ZZ)=1$.

        \item[\textup{ii)}] Probar que si $K\subseteq L$ son cuerpos de números, 
        existen infinitos primos $P$ de $\OO_K$ tales que para todo 
        primo $Q$ de $\OO_L$ con $Q\cap K = P$ se tiene que $f(Q|P)=1$.

        \textit{Sugerencia: probarlo primero para el caso en que $L/K$
        es Galois}.
    \end{enumerate}
    
    \item Probar que existen infinitos primos $p$ tal que $\bar{f}\in
    \FF_p[X]$ tiene $todas$ sus raices en $\FF_p$.
\end{enumerate}

\section*{Guía 4}

\subsection*{Ejercicio 4} Sea $p$ un primo impar.
\begin{enumerate}[a)]
    \item Probar que existen $r,s\in\ZZ$ tales que $r^2+s^2+1\equiv0 \text{ (mod } p)$.
    
    \item Para $r,s$ como en el item anterior, consideremos el reticulo
    
    $L=\lc (p,0,0,0),(0,p,0,0),(r,s,1,0),(s,-r,0,1) \rc_\ZZ \subseteq \ZZ^4$.
    \begin{enumerate}
        \item[\textup{i)}] Probar que si $v\in L$ entonces $||v||_2^2\equiv 0\text{ (mod } p)$.

        \item[\textup{ii)}] Calcular vol$(L)$.
    \end{enumerate}
    
    \item Probar que $p$ es suma de cuatro cuadrados en $\ZZ$. 
\end{enumerate}

\textit{Nota: usando que la norma en el álgebra de cuaterniones de Hamilton es multiplicativa, de (c) se sigue el teorema de Lagrange: todo $n\in\NN$ es suma de cuatro cuadrados en $\ZZ$}.


\subsection*{Ejercicio 11} Sea $d<0$ un entero libre de cuadrados. Denotemos por $h_d$ al numero de clases de $\QQ(\sqrt{d})$.

Probar que si $d$ es par y $3 \nmid h_d$, entonces la ecuacion
$$y^2=x^3+d$$
tiene a lo sumo una solucion $(x,y)\in\ZZ\times\NN$.

\section*{Guía 5}



\newpage

\section*{Soluciones}

\subsection*{Guía 1, ejercicio 4}
Un poco hablemos de la solución del ejercicio que presenta el resultado
teorico para explicar el cálculo. Primeramente sabemos que como 
$R \subsetneq \OO_K$ es un orden (por definicion subanillo de $\OO_K$
de rango máximo) entonces cumple que $\Delta_R = \Delta_K [\OO_K:R]^2$, y podemos usar ese cuadrado para nuestro favor. 

Teoricamente, si el indice es mayor a 1, habrá un elemento no trivial en el cociente, que corresponderá a un elemento en $\OO_K$ y no en $R$, y seguro de orden $p$ para un primo que divida a $[\OO_K:R]$ por el teorema de Cauchy. Podemos buscar ese primo entre los que al cuadrado dividen al valor de $\Delta_R$; y si justo $p^2$ dividia a $\Delta_K$ mala suerte, lo descartamos y seguimos probando.

Para pasar del resultado teorico existencial de un elemento no trivial en $\OO_K/R$ a construir un elemento en $\OO_K$ y no en $R$ notamos que es un $\alpha \in \OO_K - R$ tal que $p\alpha \in R$, asi que podemos iterar por los elementos de la forma $\beta /p$ con $\beta\in R$ hasta que alguno pertenezca a $\OO_K$. Este proceso será finito pues consiste en variar los coeficientes de una combinacion lineal de los elementos de la base de $R$ por los valores entre $1$ y $p$, ya que pasarnos de eso es simplemente sumar un elemento de $R$ (un multiplo entero de un coeficiente de la base).

Una vez dicho eso, el \hyperref[codigo1]{código} es bastante directo. Estamos usando el orden generado por las potencias del generador de la extension, que es una forma que no falla en crearnos un orden dentro de $\OO_K$, que luego iremos agrandando hasta llegar a todo el anillo de enteros. Probamos con el cuerpo $\QQ(\sqrt{-11}, \sqrt{13}) = Desc(x^4-x^2+36)$ cuyo tag en lmfdb es 4.0.20449.1, porque es el primero que aparece con primos 2 y 3 ``inesenciales'' (dividen a $\Delta_R$ y no a $\Delta_K$) para iterar varias veces el bloque for. Tambien con este cuerpo vimos que el codigo tardo mucho mas de lo que deberia porque despues de encontrar el anillo de enteros siguio iterando por el resto de primos (11 y 13 porque sus cuadrados dividian al discriminante).



\subsection*{Guía 1, ejercicio 15}
Los conjugados de $\zeta$ son sus potencias, entonces calcular la norma
de $\zeta^2+\zeta^3$ va a ser multiplicar lo siguiente:
$$(\zeta^2+\zeta^3)(\zeta^4+\zeta^6)(\zeta^6+\zeta^9)(\zeta^8+\zeta^{12})
= (\zeta^2+\zeta^3)(\zeta^4+\zeta^1)(\zeta^1+\zeta^4)(\zeta^3+\zeta^2)=$$
$$=\lp(\zeta^2+\zeta^3)(\zeta^4+\zeta^1)\rp^2=
(\zeta^6+\zeta^3+\zeta^7+\zeta^4)^2=(\zeta^1+\zeta^3+\zeta^2+\zeta^4)^2=
1$$
porque $\zeta$ es raiz de $X^4+X^3+X^2+X+1 = \frac{X^5-1}{X-1}$.

Como ahora sabemos que $\zeta^2+\zeta^3$ tiene norma 1 y es suma de 
enteros algebraicos pues (de vuelta) $\zeta$ es raiz de un polinomio con
coeficientes enteros, entonces pertenece a las unidades del anillo de 
enteros. Pero tambien como vemos que su valor no es ni 0, ni 1, ni -1 y 
es real (geometricamente porque $\bar{\zeta^2}=\zeta^{-2} = 
\zeta^3$ o sino porque 
$(\zeta^2+\zeta^3)^2=\zeta^4+\zeta^1+2$ entonces $\zeta^2+\zeta^3$ 
será raiz de $X^2+X-1$ que es el famoso $\bar{\varphi}$ que cae en 
$\QQ[\sqrt{5}]\subseteq \RR$) entonces todas sus potencias serán 
distintas y tendrán la misma norma, dando lugar a infinitos 
elementos distintos en $\OO_K^\times$.


\subsection*{Guía 2, ejercicio 4}

El teorema chino del resto dice que si $\aaa = \aaa_1 \cdots \aaa_r$ 
son ideales coprimos (en 
nuestro caso seran $\pp_i^{e_i+1}$ primos distintos) entonces 
$$\OO / \aaa \cong \bigoplus_{i=1}^r \OO / \aaa_i$$
y en la demostracion se prueba que el morfismo $\varphi:\OO\to
\oplus_{i=1}^r \OO / \aaa_i$ dado por $\lp\varphi(\alpha)\rp_i=\alpha\mod 
\aaa_i$ tiene nucleo $\aaa$ y es sobreyectivo.

Para resolver el problema es suficiente probar que existe un elemento 
$\alpha \in \OO$ tal que $\alpha \equiv0\mod \pp_i^{e_i}$ pero 
$\alpha \not\equiv 0 \mod \pp_i^{e_i+1}$ 
simultaneamene para todo $i$ y la gran ventaja que da el teorema es que 
basta verlo en cada coordenada y luego tomar preimagen. Ahora como 
$\pp_i^{e_i+1} \subsetneq \pp_i^{e_i}$ (son distintos porque estan 
factorizados
como producto de primos y sus factorizaciones son distintas), entonces 
existen elementos $\alpha_i \in \pp_i^{e_i} - \pp_i^{e_i+1}$ para todo $i$.
Entonces tomamos $\alpha \in \bigoplus_{i=1}^r \OO / \pp_i^{e_i+1}$ tal que
$(\alpha)_i = \bar\alpha_i\mod \pp_i^{e_i+1}$ y el elemento $\varphi^{-1}
(\alpha)$ cumplirá lo que queremos (cualquier representante en $\OO$).


\subsection*{Guía 2, ejercicio 5}

Queremos ver que dado un ideal $\aaa = \pp_1^{e_1}\cdots\pp_n^{e_n}$ 
esta generado por un solo elemento $\alpha$. Vamos a usar la 
definicion de dominio de Dedekind que nos dice que todo ideal se 
factroriza de forma unica como producto de ideales primos.
Como los ideales primos son finitos vamos a escribirlos como $\pp_1$
hasta $\pp_n$ de modo que en las factorizaciones aparezcan todos (los
exponentes pueden ser $0$).

Ahora tomo $\alpha$ como el que nos da el ejercicio anterior, con las 
valuaciones iguales a la factorizacion de nuestro ideal. Por tener 
factorizacion unica en ideales primos entonces $(\alpha)$ se escribirá
de una unica forma como $\pp_1^{f_1}\cdots\pp_n^{f_n}$.

Por ser $\OO$ un Dedekind vale que el producto de ideales primos es la 
interseccion, y como $\alpha \in \pp_i^{e_i}$ (por definicion de valuacion)
para todo $i$ entonces $\alpha \in \aaa$ con lo que $(\alpha)\subseteq\aaa$.
Es decir $f_i\geq e_i$. Pero como $\alpha \notin \pp_i^{e_i+1}$ entonces 
debe ser que $f_i=e_i$ para todo $i$.


\subsection*{Guía 3, ejercicio 6}

Probemos previamente un lema que vamos a estar usando todo el ejercicio: si $n$ y $m$ son coprimos libres de cuadrados entonces $\QQ(\sqrt{n}) = \QQ(\sqrt{m})$ si y solo si $n=m$. La vuelta es obvia y la ida la demostraremos por contrarreciproco: si $n \neq m$ y los cuerpos son iguales entonces $\sqrt{n} = a+b\sqrt{m}$. Tenes que $a\neq 0$ porque sino $\sqrt{n} = b\sqrt{m} = \sqrt{b^2m}$ no es libre de cuadrados, y que $b\neq 0$ porque sino $\sqrt{n} = a$ te queda $n=a^2$ no libre de cuadrados. Con eso, elevando al cuadrado vemos que $n = a^2+b^2m+ab\sqrt{m}$, osea $n-a^2-b^2m = ab\sqrt{m}$ donde el lado izquierdo es racional y el derecho irracional, llegando a un absurdo.

Las tres subextensiones cuadraticas de $\QQ(\sqrt{n},\sqrt{m})$ son $\QQ(\sqrt{n})$, $\QQ(\sqrt{m})$ y $\QQ(\sqrt{nm})$. Recordemos que en las extensiones cuadraticas tenemos completamente caracterizado el comportamiento de la ramificacion de los primos con el simbolo de Legendre: En $\QQ(\sqrt{D})$
\begin{itemize}
    \item Si $\leg{D}{p}=0$, es decir $p|D$, entonces ramifica: $p\OO_K = \pp^2$.
    \item Si $\leg{D}{p}=1$ entonces se parte: $p\OO_K = \pp\bar{\pp}$ (ideales distintos).
    \item Si $\leg{D}{p}=-1$ entonces es inerte: $p\OO_K$ es primo.
\end{itemize}

Ahora, como $n$ y $m$ son coprimos vamos a poder fabricarnos los primos que pide el enunciado con facilidad viendo como se comportan $\leg{n}{p}$ y $\leg{m}{p}$ por separado:

Para conseguir un primo que se parta completamente en $K$ podemos tomar un $p$ tal que $\leg{n}{p} = \leg{m}{p} = 1$ y entonces $p\OO_{\QQ(\sqrt{n})} = \pp \bar{\pp}$ y $p\OO_{\QQ(\sqrt{m})} =\pp' \bar{\pp}'$. En particular si un primo no ramifica en dos extensiones entonces no va a ramificar en la composicion, y con eso ya nos aseguramos de que $p$ se parta completamente en $\QQ(\sqrt{n},\sqrt{m})$. Esta proposicion la probamos en la teorica. Un ejemplo puede ser el numero 11 en la extension $\QQ(\sqrt{3},\sqrt{5})$, que se factoriza como:
$$\lp-\frac{3}{2}+\sqrt{3}-\frac{\sqrt{5}}{2}\rp
\lp\frac{5}{2}+\sqrt{3}-\frac{\sqrt{5}}{2}\rp
\lp\frac{1}{2}-\frac{\sqrt{3}}{2}+\frac{\sqrt{5}}{2}+\frac{\sqrt{15}}{2}\rp
\lp\frac{5}{2}-2\sqrt{3}+\frac{3\sqrt{5}}{2}-\sqrt{15}\rp$$

Para conseguir un primo que se factorice como $\pp^2\pp'^2$ busquemos uno que en una extension se parta como producto de dos distintos y luego en otra lo haga como un primo al cuadrado. Para esto precisamos que divida a uno de los discriminantes y que sea un cuadrado en el otro. Un ejemplo puede ser el numero 3 en la extension $\QQ(\sqrt{3},\sqrt{7})$, que se factoriza como:
$$3= \lp 1 +\frac{\sqrt{3}}{2} + \frac{\sqrt{7}}{2}\rp^2
\lp   1 -\frac{\sqrt{3}}{2} + \frac{\sqrt{7}}{2}\rp^2$$

Para que $p$ sea inerte en $K = \QQ(\sqrt{n},\sqrt{m})$ entonces debe serlo en todas sus subextensiones, y para ello debe darse que $\leg{n}{p}=-1$, $\leg{m}{p}=-1$ y $\leg{nm}{p}=-1$. Ahora, como $\leg{n}{p}\leg{m}{p}=\leg{nm}{p}$ entonces no pueden ser los tres simultaneamente $-1$.

En el caso $D_K=4n$ se puede obviar el factor de $\leg{4}{p}$ porque vale siempre 1 para primos impares (pues el 4 es siempre $2^2$), pero esto no es cierto si el primo es igual a $2$. Para el resto de primos, para que sea totalmente ramificado entonces deberia serlo en todas las subextensiones, es decir deberia ser $\leg{n}{p}= 0$ y $\leg{m}{p}= 0$ pero eso ocurre cuando $p|n$ y $p|m$ simultaneamente pero habiamos pedido coprimalidad.

\subsection*{Guía 3, ejercicio 10}

Sea $f\in \ZZ[X]$. Si $f(0)=1$ entonces, como el coeficiente 
independiente es 1, al evaluar el polinomio en multiplos de $m$ y 
tomar módulo $m$ solo nos quedará el coeficiente independiente. Ahora,
visto en $\ZZ$ el polinomio no puede valer infinitas veces $1$ pues 
sería constante (si $n$ es el grado del polinomio entonces en $f(m), 
f(2m), \cdots, f(nm), f((n+1)m)$ debe haber alguno distinto a 1 
porque sinó $f-1$ tendria el mismo grado y mas de $n$ raices). 

Supongamos que $f$ tiene una raiz solamente en finitos primos $p_1,
\cdots, p_n$, tomemos $m=\prod p_i$ y veamos $f(km) = jm+1$ donde en
algun momento $j$ será distinto de 0, entonces $jm+1$ no va a ser 
divisible por ninguno de los $p_i$, es decir que en su factorizacion 
en primos va a aparecer algun $p_{n+1}$ nuevo, y en $\FF_{p_{n+1}}$
el polinomio tendrá una raíz.

Ahora si $f(0) = n$ no nos podemos asegurar que $f(km) = jm+n$ no 
sea un multiplo de los primos de $m$ pues puede darse que $n$ sea un
multiplo de $m$, pero esto se soluciona sacando de $m$ los primos que 
dividan a $n$ (que deberiamos incluirlos porque ahí obviamente vale 
$f(0)\equiv 0$). En resumen: defino $m$ como el producto de los primos
donde $\bar{f}$ tiene una raíz (asumiendo que son finitos) salvo 
aquellos que dividan a el coeficiente independiente, luego como 
$f(km) \equiv n \mod m$ en algun momento valdrá en $\ZZ$ que 
$f(km) = jm+n$ y ese valor tendrá factores primos distintos a los de 
$m$ (si $p|jm+n$ y $p|m$ entonces tambien $p|n$ pero esos los 
excluimos), llegando así a un absurdo pues $\bar{f}$ tendría una raíz 
en ese $\FF_p$.

Ahora pasemos al lenguaje de ramificacion de los primos. Dado $K$ un cuerpo de 
números, sabemos que podemos escribir $K=\QQ(\alpha)$ y llamemos $f$ al
polinomio minimal de $\alpha$. Tenemos la
conexión de que si nuestro polinomio $f$ se factoriza como 
producto de irreducibles módulo $p$ como $f_1^{e_1}\cdots f_r^{e_r}$ 
entonces los ideales $\pp_i = p\OO + f_i(\alpha)\OO$ son primos distintos
sobre $p$ con grado de inercia $gr(f_i)$ y $p\OO$ se factoriza como
$\pp_1^{e_1}\cdots \pp_r^{e_r}$. En particular, si $f$ tiene una raíz 
en $\FF_p$ entonces su fatorizacion módulo $p$ tendrá un componente 
lineal y eso nos dirá que el grado de inercia es 1. Por el punto $a$ esto
sucede para infinitos primos.

Si tenemos $\QQ\subseteq K\subseteq L$ entonces ya vimos que hay primos
$Q \unlhd \OO_L$ sobre un $p$ tales que $f(Q | p)=1$ y como el grado 
de inercia es multiplicativo en torres entonces 
tenemos que $f(Q | P)f(P|p)=1$ y son ambos 1. Como en las extensiones de 
Galois todos los primos sobre un determinado $P$ tienen el mismo grado 
de ramificacion y de inercia, entonces valdrá para el resto de los $Q|P$.
Luego si $K\subseteq L$ no es Galois podemos hacer el mismo razonamiento
partiendo de una $L\subseteq L'$ Galois sobre $K$ y $f(Q' | Q)f(Q | P)
f(P|p)=1$ nos dice que son todos $1$ para todo $Q'|P$ primo de $\OO_{L'}$.

Al usar la teoria ramificacion de primos en extensiones junto con Galois probamos que hay infinitos primos que son totalmente ramificados y, volviendo al lenguaje de nuestro polinomio $f$, esto nos dice que $\bar{f}\in \FF_p[X]$ se factoriza como producto de lineales, es decir tiene todas sus raices en el cuerpo.

\subsection*{Guía 4, ejercicio 4}

Si consideramos para $r,s\in \FF_p$ los conjuntos $\{r^2\}$ y $\{-s^2-1\}$ tienen ambos $\frac{p+1}{2}$ elementos distintos, asi que deben tener interseccion no vacia, que nos da una solucion a $r^2+s^2+1\equiv0 \mod p$.

Si $v\in \lc (p,0,0,0),(0,p,0,0),(r,s,1,0),(s,-r,0,1) \rc_\ZZ$ entonces sabemos que 
$$v=(\alpha p + \gamma r + \delta s , \beta p + \gamma s - \delta r ,\gamma,\delta)$$
para algunos enteros $\alpha,\beta,\gamma,\delta$. Calculemos su norma 2 viendo módulo $p$ directamente:
$$||v||_2^2 = (\alpha p + \gamma r + \delta s)^2 + (\beta p + \gamma s - \delta r)^2 +\gamma^2+\delta^2 \equiv $$
$$\equiv \gamma^2 r^2 + \delta^2 s^2 + 2\gamma r \delta s
+ \gamma^2 s^2 + \delta^2 r^2 - 2\gamma s \delta r
+\gamma^2+\delta^2 = \gamma^2(r^2+s^2+1) + \delta^2(r^2+s^2+1)$$
donde la expresion final es multiplo de $p$ porque lo era cada factor de $(r^2+s^2+1)$. 

El volumen del retículo será el determinante de la base, que como la que nos brinda el enunciado es triangular inferior, su cálculo es directo: vol$(L)=p^2$.

Consideremos la bola de centro 0 y radio $2p-\varepsilon$ en $\RR^4$, que tiene volumen $\frac{1}{2}\pi^2 (2p-\varepsilon)^2>16p^2$. El teorema de Minkowski nos dice que si en un espacio de dimension $n$ tenemos un conjunto $X$ simetrico medible y convexo (como la bola) y un reticulo $L$ tales que $\mu(X)>2^n$vol$(L)$ entonces hay algun elemento de $L$ que está en $X$ aparte del cero. Utilizando ese teorema sabemos que hay algun elemento de $L$ de norma menor a $2p$, pero como todos sus elementos tenian norma multiplo de $p$ entonces debe ser exactamente igual. Como la norma de un elemento de coordenadas enteras es una suma de cuatro cuadrados entonces encontramos una forma de escribir a $p$ como suma de cuatro cuadrados.

Si consideramos el algebra de cuaterniones de Hamilton $H=\{ a+bi+cj+dk : a,b,c,d\in \ZZ, i^2=j^2=k^2=ijk=-1 \}$ con la norma $|a+bi+cj+dk|^2=a^2+b^2+c^2+d^2$, entonces probamos que existen elementos de norma $p$ para todo primo. Como la norma es multiplicativa, entonces para cualquier $n\in \NN$ habra un elemento en $H$ de norma $n$ que podemos armar multiplicando elementos de norma $p$ para la factorizacion en primos de $n$, y entonces por ser la norma de este elemento le estamos dando a $n$ una escritura como suma de cuatro cuadrados.


\subsection*{Guía 4, ejercicio 11}




% con alt + z te acomoda las lineas


\newpage

\vspace{1em}
\noindent\textbf{Código 1.} Anillo de enteros
\label{codigo1}

\lstset{
    language=Python,
    basicstyle=\ttfamily\small,
    frame=single,
    breaklines=true
}

\begin{lstlisting}
# para chequear que hayan puesto una numberfield (nomas)
from sage.rings.number_field.number_field import NumberField_generic

# para iterar por los c_i
from itertools import product

def anillo_de_enteros(K):
    # chequeamos que sea number field
    if not isinstance(K, NumberField_generic):
        raise TypeError("K debe ser un cuerpo de numeros.")

    # Una base que nos va a dar siempre un orden en el cuerpo de numeros es la de 1, a, a^2, ... para a el generador de la extension
    # para saber cuantas potencias de a poner debemos saber el grado de la extension
    n = K.degree()
    R = K.order([a^i for i in range(n)])
    primos_que_al_cuadrado_dividen_a_dk = []
    terminamos = false
    # ahora generamos el orden con la base [1, a, ... , a^(n-1)]
    while not terminamos:
        base = R.basis()
        p = next((q for q, e in R.discriminant().factor() if e > 1 and q not in primos_que_al_cuadrado_dividen_a_dk), None)
        if p == None:
            terminamos = true
            break

        for los_c_i in product(range(int(p)), repeat=n):
            if not any(los_c_i): # saltear (0,...,0)
                continue
            alpha = sum(ai*ci for ai, ci in zip(base, los_c_i)) / p
            if alpha.is_integral():
                R = K.order(base + [alpha])
                break
        else: # no sabia que existia esto, es para cuando paso todo el for sin un break.
            primos_que_al_cuadrado_dividen_a_dk.append(p)

    return R.basis()

K.<a> = NumberField(x^4-x^2+36)
print(anillo_de_enteros(K))
print(K.ring_of_integers().basis())

\end{lstlisting}






\end{document}